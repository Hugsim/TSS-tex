\documentclass[a4paper]{article}
\usepackage{../flpack}

\begin{document}

\providecommand\fname{}
\renewcommand\fname{19-10-10}

\subsection{Mer z-transform}
Låt \(
    x[k]
\) vara en diskret signal. Då är \(
    X(z) \coloneqq \sum_{k=-\infty}^\infty x[k] z^{-k}
\).

\begin{ex}
    \(
        \delta(k-l) \xleftrightarrow{\mathcal{z}} z^{-l}
    \) 
\end{ex}

\begin{ex}
    \(
        \gamma^k u[k] \xleftrightarrow{\mathcal{z}} \frac{z}{z-\gamma} 
    \) om \(
        \abs{z} > \abs{\gamma}
    \) 
\end{ex}

\begin{ex}
    \(
        \gamma^{k-1} \xleftrightarrow{\mathcal{z}} \frac{1}{z-\gamma} 
    \) 
\end{ex}

\subsubsection{Räkneregler}
\(
    x[k+1]u[k] \xleftrightarrow{\mathcal{z}} z (X(z) - x[0])
\) 

\(
    x[k-1] \xleftrightarrow{\mathcal{z}} \inv{z} X(z)
\) 

\(
    \gamma^k x[k] \xleftrightarrow{\mathcal{z}} X(\frac{z}{\gamma})
\) 

\(
    x_1*x_2 \xleftrightarrow{\mathcal{z}} X_1 X_2
\) 

Vi börjar med lite kontext. Om \(
    f(t)
\) är en kontinuerlig signal och \(
    D = \dv{t}
\). Då är \(
    D f(t) \approx \frac{f(t+h) - f(t)}{h} 
\) för små \(
    h
\). Vi definierar \(
    x[k] \coloneqq f(hk)
\). Då är \(
    D \approx \frac{1}{h} (E-1)
\) där \(
    Ex[k] = x[k+1]
\). En differentialekvation är på formen \(
    Q(D)f = P(D)g
\) vilket genom approximering som ovan ger en differensekvation \(
    Q\qty(\frac{E-1}{h} ) x_f = P\qty(\frac{E-1}{h} ) x_g
\).

\subsection{Frekvenssvar}
Låt \(
    y = h*x
\) vara ett LTI-system. Förra gången såg vi att \(
    h * z^k = H(z) \cdot z^k
\). Om vi tar \(
    z = e^{\iO}
\) och \(
    x[k] = e^{\iO k}
\) får vi \(
    h * x[k] = H(e^{\iO k}) x[k]
\). Vi brukar kalla \(
    H
\) i den ekvationen för frekvenssvar.

\subsubsection{Reell form}
Låt \(
    x[k] = \cos (\Omega k + \beta)
\) där \(
    \Omega, \beta \in \realn
\). Då är \(
    h*x[k] = \half h*e^{i(\Omega k + \beta)} + \half h*e^{-i(\Omega k + \beta)}
        = \half \qty( H(e^{\iO }) e^{j(\Omega k + \beta)} + H(e^{-\iO}) e^{-j(\Omega k + \beta)} )
        = \{ \text{om } h \text{ reell är } H(\overline{z}) = \overline{H(z)} \}
        = \half \qty( H(e^{\iO }) e^{j(\Omega k + \beta)} + \overline{H(e^{-\iO}) e^{-j(\Omega k + \beta)} })
        = \Re(\abs{H(e^{\iO})} e^{i(\Omega k + \beta) + i \angle H(e^\iO)})
        = \abs{H(e^\iO)} \cos(\Omega k + \beta + \angle H(e^\iO))
\).

\subsubsection{Koppling till DFT}
Om \(
    x[k]
\) är en \(
    N
\)-periodisk och kausal signal, d.v.s.\ \(
    x[k+N] = x[k] \; \forall \; k \geq 0
\) och \(
    x[k] = 0
\) för \(
    k < 0
\).

Då är \(
    Fx[k] = \sum_{k=0}^{N-1} x[k] e^{-i \frac{2\pi}{N} kl}
\).

\(
    X(z) = \sum_{k=0}^\infty x[k] z^{-k}
         = \{ k = nN+m \}
         = \sum_{n=0}^\infty \sum_{m=0}^{N-1} x[m] z^{-nN-m}
         = \sum_{n=0}^\infty z^{-nN} \sum_{m=0}^{N-1} x[m]z^{-m}
         = \{\text{Geometrisk summa}\} 
         = \frac{1}{1-z^{-N}} \cdot \sum_{m=0}^{N-1} x[m]z^{-m}
\). Alltså är \(
    Fx[l] = (1-z^{-N})X(z)
\) för \(
    z = e^{i \frac{2\pi}{N} l}
\).

\subsection{Differensekvationer}
Låt \(
    Q = z^n + a_{n-1} z^{n-1} + \cdots + a_1 z + a_0
\), \(
    P = b^n z^n + b_{n-1} z^{n-1} + \cdots + b_1 z + b_0
\) och \(
    Ey[k] = y[k+1]
\). 

Differensekvationer kan i allmänhet skrivas \(
    \left\{\begin{matrix}
        Q(E) y[k] = P(E)x[k] & k \geq 0 \\ 
        x \text{ given och } y[k] \text{ givna för } k = -n, -n+1, \dots, -2, -1 & ~
    \end{matrix}\right.
\).

\(
    Q(E)y = P(E)x \iff y[k] + a_{n-1}y[k-1] + \cdots + a_1 y[k-n+1] + a_0 y[k-n]
          = b_n x[k] + b_{n-1} x[k-1] + \cdots + b_1 x[k-n+1] + b_0 x[k-n]
\) vilken vi delar upp i två ekvationer, zero-state (partikulärlösning) (\(
    Q(E)y_0[k] = P(E)x[k]
\) där \(
    y_0[k] = 0
\)) och zero-input (homogenlösning) (\(
    Q(E) y_{in}[k] = 0
\) där \(
    y_{in}[k] = y[k]
\)). 

\begin{ex}
    Vi börjar med att lösa ett exempel på zero-input.

    Låt \(
        Q(z) = z^2+a_1z+a_0
    \), d.v.s.\ vi vill lösa \(
        y[k] + a_1y[k-1]+a_0y[k-2] = 0
    \) och vi har \(
        y[-2], y[-1]
    \) givna.

    Idén är att använda z-transformen.

    differensekvationen ovan ger \(
        0   = \sum_{k=0}^\infty (y[k] + a_1y[k-1] +a_0y[k-2]) z^{-k}
            = \sum_{k=0}^\infty y[k]z^{-k} 
                + \sum_{k=1}^\infty a_1y[k-1] z^{-k}
                + \sum_{k=2}^\infty a_0y[k-2] z^{-k}
                + a_1y[-1] z^0 + a_0y[-2]z^0 + a_0 y[-1] z^{-1}
            = \sum_{k=0}^\infty y[k]z^{-k} 
                + \sum_{k=0}^\infty a_1y[k] z^{-k-1}
                + \sum_{k=0}^\infty a_0y[k] z^{-k-2}
                + a_1y[-1] z^0 + a_0y[-2]z^0 + a_0 y[-1] z^{-1}
            = Y(z) + a_1 \inv{z} Y(z) + a_0 \inv{z^2} Y(z)
            = \inv{z^2} \qty( Q(z) Y(z) + (a_1y[-1] + a_0y[-2]) z^2 + a_0y[-1]z)
    \) vilket ger att \(
        Y(z) = \inv{Q(z)} ((a_1y[-1] + a_0y[-2]) z^2 + a_0y[-1]z)
             = \frac{\alpha z^2 + \beta z}{Q(z)} 
    \) (Där \(
        \alpha = a_1y[-1] + a_0y[-2]
    \) och \(
        \beta = a_0y[-1]
    \)) \(
        = \frac{\alpha z^2+\beta z}{(z-z_+) (z-z_-)} 
    \). Lösningen fås genom att ansätta \(
        y[k] = 
        \left\{\begin{matrix}
            Az_+^k + Bz_-^k & \text{om } z_+ \neq z_-  \\ 
            Az_+^k + Bkz_+^k & \text{om } z_+ = z_- 
        \end{matrix}\right.
    \). \(
        A
    \) och \(
        B
    \) hittas genom \(
        Az_+^{-1} + Bz_-^{-1} = y[-1]
    \) och \(
        Az_+^{-2} + Bz_-^{-2} = y[-2]
    \).

    (\(
        z_+
    \) och \(
        z_-
    \) är lösningar till \(
        Q(z) = 0
    \))

    Nu löser vi zeroinput för \(
        Q(z) = z^2 + a_1z + a_0
    \). För att få fram lösningen utvidgar vi \(
        x 
    \) och \(
        y_0
    \) till kausala signaler så att \(
        Q(E)y = P(E)x
            \implies Y_0(z) \qty( 1 + \frac{a_{n-1}}{2} + \frac{a_{n-2}}{2} + \cdots + \frac{a_0}{2} ) 
                = X(z) \qty( b_n + \frac{b_{n-1}}{2} + \frac{b_{n-2}}{2} + \cdots + \frac{b_0}{2} ) 
    \) vilket om vi multiplicerar med \(
        z^n
    \) ger \(
        Q(z)Y_0(z) = P(z)X(z)
    \), d.v.s.\ \(
        Y_0(z) = H(z)X(z)
    \) där \(
        H(z) = \frac{P(z)}{Q(z)} 
    \) \(
        \iff y_0[k] = h*x[k]
    \) där \(
        h
    \) har z-transform \(
        H
    \).

    \(
        H(z)
    \) är en rationell funktion, d.v.s.\ kvoten av två polynom, vilket ger
    att man får partialbråksuppdela och algebra-a sig till termer vi vet 
    hur man invers-z-transformerar.

    \begin{påst}
        Om \(
            H(z) = \frac{P(z)}{Q(z)} 
        \) där \(
            \text{deg}(P) \leq \text{deg}(Q) = n 
        \) så är \(
            h[k] = b_n \delta[k] + \sum_{m=1}^M \sum_{l=1}^{N_m} c_{l,m} k^l z_m^k
        \) där \(
            z_1, z_2, \dots, z_m
        \) är nollställen till \(
            Q
        \) och \(
            c_{l,m}
        \), \(
            M
        \) och \(
            N_m 
        \) är godtyckliga tal.
    \end{påst}

    \begin{sats}[Differensekvationer och stabilitet]
        Som en följd av det är differensekvationen \(
            Q(E)y[k] = P(E)x[k]
        \) där \(
            k \geq 0
        \) och \(
            y[k]
        \) kända då stabil, d.v.s.\ \(
            y[k] \xrightarrow{k\to \infty} 0
        \) när \(
            x[k] \xrightarrow{k \to \infty} 0
        \) om och endast om \(
            Q(z) = 0    
        \) saknar lösning med \(
            \abs{z} \geq 1
        \).
    \end{sats}
    
\end{ex}

\begin{sats}[Stabilitet och väldefinierade system]
    Ett diskret system är stabilt \(
        \iff 
    \) \(
        H(z)
    \) väldefinierat för \(
        \abs{z} \geq 1
    \). Det är samma sak som att \(
        H
    \) saknar poler i \(
        \abs{z} \geq 1
    \).
\end{sats}

\subsection{Poler till överföringsfunktioner}
\begin{defn}[Poler]
    Låt \(
        H(z) = \frac{P(z)}{Q(z)} 
    \) vara en rationell funktion på reducerad form, d.v.s.\ \(
        P
    \) och \(
        Q
    \) saknar gemensamma nollställen. 

    \(
        z_0 \in \compn
    \) sägs vara en pol till \(
        H
    \) om \(
        Q(z_0) = 0
    \).
\end{defn}

\begin{ex}
    Låt \(
        P(z) = b_1z + b_0
    \) och \(
        Q(z) = z^2+a_1z+a_0
    \). Anta också att \(
        Q(\frac{-b_0}{b_1}) \neq 0
    \) och att \(
        a_0 \neq \frac{a_1^2}{4} 
    \) vilket ser till att \(
        P
    \) och \(
        Q
    \) inte är noll samtidigt och att \(
        z_- \neq z_+
    \).

    Då är \(
        H(z) = \frac{P(z)}{Q(z)} = \frac{b_1z+b_0}{(z-z_+)(z-z_-)} 
    \) där polerna till \(
        H
    \) är \(
        z_- 
    \) och \(
        z_+
    \). Vad är då \(
        h
    \), d.v.s.\ vad är \(
        z^{-1}(H)
    \)?

    Vi skriver \(
        H(z) = \frac{b_1(z-z_+) + b_1 z + b_0}{(z-z_+)(z-z_-)}
             = \frac{b_1}{z-z_-} + \frac{b_1z_+  +b_0}{(z-z_+)(z-z_-)}
             = \{\text{Partialbråksuppdelning}\}
    \) så att \(
        h[k] = \qty(\frac{b_1z_-+b_0}{z_--z_+} z_-^{k-1} + \frac{b_1z_++b_0}{z_+-z_-} z_+^{k-1}) u[k-1]
    \).

    Vi får alltså att zerostate-lösningen för \(
        y[k +2] + a_1y[k+1]+a_0y[k] = b_1x[k+1]+b_0x[k]
    \) där \(
        y[-2] = y[-1] = 0
    \) ges av \(
        y = h*x
    \) där \(
        h
    \) är som ovan.

    Stabilitet \(
        \iff \abs{z_+}, \abs{z_-} < 1
    \).
\end{ex}

\end{document}