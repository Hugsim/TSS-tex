\documentclass[a4paper]{article}
\usepackage{../flpack}

\begin{document}

\providecommand\fname{}
\renewcommand\fname{19-10-01}

\dquote{Bupp bupp bupp}{Mattemannen}
\subsection{Laplace och differentialekvationer}
\begin{ex}
    Vi har en RLC-krets:

    \f

    Vi är intresserade av strömmen \(
        y(t)
    \) som går genom kretsen.

    Kirchoff ger att \(
        V_L(t) + V_R(t) + V_C(t) = x(t)
    \). Vi vet också att \(
        V_L(t) = L \cdot y'(t)
    \), \(
        V_R(t) = R \cdot y(t)
    \) och \(
        V_C(t) = \inv{C} \int_{-\infty}^t y(\tau) \dd \tau
    \).

    Därmed har vi \(
        Ly' + Ry + \inv{C} \int_{-\infty}^t y(\tau) \dd \tau = x(t)
    \) eller genom att derivera en gång till \(
        L y'' + Ry' + \inv{C} y = x' 
            \iff Q(D) y = P(D) x
    \) där \(
        Q(D) = D^2 + \frac{R}{L} D + \inv{LC} 
    \) och \(
        P(D) = \inv{L} D
    \). Detta är en andra ordningens differentialekvation.

    Vi kommer tillbaka till hur vi löser detta, men först tittar vi på hur vi
    löser enklare fall.
\end{ex}

Om \(
    P(D) = 1
\) kan vi använda \(
    y = y_0 + y_{\text{in}}
\) där zerostate \(
    y_0
\) löser \(
    \left\{\begin{matrix}
        Q(D)y_0 = x\\  
        y_0^{(j)}(0) = 0 \; \forall \; j 
    \end{matrix}\right.
\) och zeroinput \(
    y_{\text{in}} 
\) löser \(
    \left\{\begin{matrix}
        Q(D)y_{\text{in}} = 0\\ 
        y_\text{in}^{(j)}(0) = y^{(j)}(0) \; \forall \; j 
    \end{matrix}\right.
\).

Vi börjar med att lösa zerostate. Vi vet att \(
    \lap(Q(D)y_0) = X(s)
\) och \(
    \lap(D^2y_0) = s^2\lap(y)(s)
\). Därmed har vi att \(
    Y_0(s) = \inv{Q(s)} X(s) 
        \iff y_0(t) = h*x(t)
\) där \(
    h
\) uppfyller att \(
    H(s) = \inv{Q(s)}
\).

\begin{ex}
    \(
        Q(D) = D + a
    \), d.v.s.\ vi tittar på \(
        y'+ay = x
    \). Då är \(
        Q(s) = s+a
    \) så att \(
        H(s) = \inv{s+a} = \lap(e^{-at}u(t))(s)
    \) och därmed är \(
        h(t) = e^{-at}u(t)
    \) och \(
        y_0(t) = h*x(t) = \int_0^t e^{-a(t-\tau)}x(\tau)\dd \tau
    \).
\end{ex}

\begin{ex}
    Låt \(
        Q(D) = D^2 + aD + b
    \). Då är \(
        Q(s) = s^2 + as + b
    \) och därmed \(
        H(s) = \inv{s^2 + as + b}
    \). Viktigt här (men även generellt) är partialbråksuppdelning.
    
    I det här fallet vill vi hitta \(
        \lambda_{\pm}
    \) som löser \(
        \lambda^2 + a \lambda + b = 0
    \). Då är \(
        s^2 + a s + b = (s-\lambda_+)(s-\lambda_-)
    \).

    Vi vill förenkla \(
        \inv{(s-\lambda_+)(s-\lambda_-)}
    \) och använder partialbråksuppdelning för det. Då får vi att det 
    är lika med \(
        \inv{\lambda_+ - \lambda_-} \left( \inv{s-\lambda_+} - \inv{s-\lambda_-} \right)
    \). 

    Då är \(
        \inv{s^2 + as + b} = \left\{\begin{matrix}
            \inv{\lambda_+ - \lambda_-} \left( \inv{s-\lambda_+} - \inv{s-\lambda_-} \right) & \lambda_+ \neq \lambda_- \\ 
            \inv{(2-\lambda_+)^2} & \lambda_+ = \lambda_-
        \end{matrix}\right.
    \) vilket då när vi använder Laplace ger att det är lika med \(
        \left\{\begin{matrix}
            \frac{u(t)}{\lambda_+ - \lambda_-} \left( e^{\lambda_+ t} - e^{\lambda_- t} \right) & \lambda_+ \neq \lambda_- \\ 
            t e^{\lambda_+ t} u(t) & \lambda_+ = \lambda_-
        \end{matrix}\right.
    \).
\end{ex}

Vi går sedan vidare till att lösa zeroinput: Då vill vi lösa \(
    \left\{\begin{matrix}
        Q(D)y_{\text{in}} = 0 \\
        y_{\text{in}}^{(j)} (0) = y^{(j)}(0) \; \forall \; j
    \end{matrix}\right.
\). 

\begin{ex}
    Låt \(
        Q(D) = D^2 + aD + b
    \). Då är \(
        \lap(Q(D) y_{\text{in}}) = \lap(D^2 y_{\text{in}}) + a \lap(D y_{\text{in}}) + b \lap(y_{\text{in}})
            = \lap(y_{\text{in}}) - s y(0) - y'(0) + as\lap(y_{\text{in}}) - a y(0) + b \lap(y_{\text{in}})
            = (s^2+as+b)\lap(y_{\text{in}}) - y'(0)-(s+a)y(0)
    \) vilket med antaganden som gjordes för zeroinput ger att \(
        \lap(y_{\text{in}})(s) = \frac{y'(0)+(s+a)y(0)}{Q(s)} 
    \). Det löses på ett liknande sätt som det andra, med 
    partialbråksuppdelning.

    \begin{align*}
        \lap(y_{\text{in}})(s) &= \frac{y'(0)+(s+a)y(0)}{Q(s)} \\
            &= \frac{y'(0) + (\lambda_+ + a) y(0) + (s-\lambda_+)y(0)}{(s-\lambda_+)(s-\lambda_-)} \\
            &= (y'(0)+(\lambda_+ + a)y(0))H(s) + \frac{y(0)}{s-\lambda_-} \\
            &= \lap( (y'(0) + (\lambda_+ + a)y(0))h(t) + y(0) e^{\lambda_- t} u(t))
            &\iff y_{\text{in}}(t) = (y'(0) + (\lambda_+ + a)y(0))h(t) + y(0) e^{\lambda_- t} u(t)
    \end{align*}
\end{ex}

Allmänt, om \(
    Q(D)y_{\text{in}} = 0
\) är \(
    Q(s) y_{\text{in}}(s) = \sum_{k=0}^{n-1} g_{Q,k}(s)y^{(k)}(0)
\) där \(
    g_{Q,k} 
\) ges av att \(
    g_{Q,n-1}(s) = 1
\), \(
    g_{Q,n-2}(s) = s + a_{n-1}
\), \(
    g_{Q,n-3}(s) = s^2 + a_{n-1} s + a_{n-2}
\), \dots, \(
    g_{Q,0}(s) = s^{n-1} + a_{n-1} s^{n-2} + \cdots + a_2 s + a_1
\).

Då är \(
    Y_m(s) = \frac{\sum_{k=0}^{n-1} g_{Q,k}(s)y^{(k)}(0)}{Q(s)} 
\). 

\dquote{Låt oss ta ett ruggigt konkret exempel.}{Goffeng}

\dquote{Förr, när det var bättre.}{Goffeng}

\subsubsection{Mer allmänt} 
För en differentialekvation \(
    Q(D) y = P(D) x
\) med givet \(
    x
\) och \(
    y^{(j)}(0)
\) givna för relevanta \(
    j
\) får vi av Laplace \(
    Q(s) Y(s) - \sum_{k=0}^{n-1} g_{Q,k}(s) y^{(k)}(0) 
        = P(s) X(s) - \sum_{k=0}^{n-1} g_{P,k}(s) x^{(k)}(0)
\) och därmed \(
    Y(s) = H(s)X(s) + \inv{Q(s)} \sum_{k=0}^{n-1} \left( g_{Q,k}(s) y^{(k)}(0) - g_{P,k}(s) x^{(k)}(0) \right)
\) där \(
    H(s) = \frac{P(s)}{Q(s)} 
\) är överföringsfunktionen. Detta ger att \(
    y(t) = h*x(t) + \sum_{k=0}^{n-1} \left( \alpha_{Q,k}(t) y^{(k)}(0) - \beta_{P,k}(t) x^{(k)}(0) \right)
\) där \(
    h = \lap^{-1}(H)
\), \(
    \alpha_{Q,k} = \lap^{-1}(\frac{g_{Q,k}}{Q} )
\) och \(
    \beta_{P,k} = \lap^{-1}(\frac{g_{P,k}}{Q} )
\).

Hur hittar vi \(
    h
\), \(
    \alpha_{Q,k}
\) och \(
    \beta_{P,k}
\)? 

Båda handlar om att inverstransformera uttryck på formen \(
    \frac{s^m}{Q(s)} 
\) där \(
    m \leq n
\).

\begin{påst}
    Om vi kan faktorisera \(
        Q(s) = (s-\lambda_1)^{\gamma_1}(s-\lambda_2)^{\gamma_2}\cdots (s-\lambda_p)^{\gamma_p}
    \) så är \(
        \frac{s^m}{Q(s)} 
    \) en summa av termer på formen \(
        \frac{s^l}{(s-\lambda_j)^{\gamma_j}} 
    \) där \(
        l < \gamma_p
    \) eller \(
        l = 1
    \).
\end{påst}

\begin{påst}
    För \(
        h = \lap^{-1}(\frac{P}{Q}) 
    \) gäller att för några konstanter \(
        l_k
    \) och \(
        c_{j,k}
    \) är \(
        \sum_{k=1}^P \sum_{i=1}^{l_k} \left( c_{i,k} t^i e^{\lambda_k t} u(t)\right) + b_n \delta(t)
    \). 
\end{påst}

Som följd av dessa påståenden får vi att \(
    h
\) är stabil (\(
    \int \abs{h(t)} \dd t < \infty
\)) om och endast om \(
    Q(\lambda) = 0 
\) saknar lösningar med \(
    \Re{\lambda} \geq 0
\).

\begin{påst}
    Om \(
        h
    \) är stabil gäller att 
    \begin{enumerate}
        \item Om \(
            y
        \) löser \(
            y = h*x
        \) följer att \(
            x 
        \) begränsad \(
            \implies \; y
        \) begränsad.
        \item Om \(
            y
        \) löser \(
            y = h * x
        \) följer att \(
            \int_0^\infty \abs{x(t)} \dd t < \infty \implies \int_0^\infty \abs{y(t)} \dd t < \infty
        \).
    \end{enumerate}

    \begin{proof}
        ~ \\
        \begin{enumerate}
            \item Om \(
                \abs{x(t)} \leq C \; \forall \; t
            \) gäller att \(
                \abs{y(t)} = \abs{ \int_0^t h(t-\tau) x(\tau) \dd \tau } 
                    \leq \int_0^t \abs{h(t-\tau)} \abs{x(\tau)} \dd \tau
                    \leq C \int_0^t \abs{t-\tau} \dd \tau
                    \leq C \int_0^\infty \abs{h(\tau)} \dd \tau
            \).

            \item Bevisas inte här, finns i boken.
        \end{enumerate}
    \end{proof}
\end{påst}

\begin{ex}
    Lös \(
        y''(t) + ay'(t) + by(t) = cx'(t) + dx(t)
    \) där \(
        y(0)
    \), \(
        y'(0)
    \) och \(
        x(t)
    \) är givna.

    Då har vi att \(
        \lap(y''+ay'+by) = (s^2 + as + b) Y - y'(0) - (s+a)y(0)
    \) och \(
        \lap(cx'+dx) = (cs+d)X - cx(0)
    \). Därmed är \(
        (s^2+as+c) Y(s) - y'(0) - (s+a)y(0) = (cs+d)X - cx(0)
    \) \(
        \implies 
    \) \(
        Y(s) = \frac{cs+d}{s^2+as+b} X(s) + \frac{y'(0)+(s+a)y(0)-cx(0)}{s^2+as+b} 
    \).

    Vi vet att \(
        H(s) = \frac{cs+d}{s^2+as+b} 
    \). Vad är då \(
        h
    \)? Vi skriver \(
        s^2+as+b = (s+\frac{a}{2} )^2 + b - \frac{a^2}{4} 
    \). Då är \(
        H(s) = \frac{cs+d}{(s+\frac{a}{2} )^2 + b - \frac{a^2}{4}}
             = \frac{c(s+\frac{a}{2} -\frac{ac}{2} + d)}{(s+\frac{a}{2} )^2 + b - \frac{a^2}{4}}
             = c \frac{s+\frac{a}{2}}{(s+\frac{a}{2} )^2 + b - \frac{a^2}{4}} + \frac{d-\frac{ac}{2})}{(s+\frac{a}{2} )^2 + b - \frac{a^2}{4}}
             = \lap\left(c e^{-\frac{a}{2} t} \cos(\sqrt{b - \frac{a^2}{4} } t)\right) + \lap\left( \frac{d-\frac{ac}{2} }{\sqrt{b-\frac{a^2}{4} }} e^{-\frac{a}{2} t} \sin( \sqrt{b - \frac{a^2}{4} } t) \right)
    \).
\end{ex}

\end{document}