\documentclass[a4paper]{article}
\usepackage{../flpack}

\begin{document}

\providecommand\fname{}
\renewcommand\fname{19-09-19}

\subsection{Fourierserier och -transformer}
\begin{påm}
    Kom ihåg Eulers formel: \(
        \cos(x) = \half (e^{ix} + e^{-ix})
    \). Kom också ihåg den allmänna formeln: \(
        x(t) = A \cos(\omega t + \theta)
        = \frac{A}{2} \left( e^{i(\omega t + \theta)} + e^{-i(\omega t + \theta)} \right)
        = \frac{A}{2} e^{i\theta} e^{i\omega t} + \frac{A}{2} e^{-i\theta} e^{-i\omega t}
    \). Om \(
        \omega = k \oz
    \) kan vi skriva \(
        x(t) = \frac{A}{2} e^{i\theta} e^{ik\oz t} + \frac{A}{2} e^{-i\theta} e^{-ik\oz t}
    \) där vi ger namnen \(
        c_k = \frac{A}{2} e^{i\theta}
    \) och \(
        c_{-k} = \frac{A}{2} e^{-i\theta}
    \). Allmännt för en periodisk signal med flera sinsuformade signaler med \(
        \omega = t\oz
    \) får vi \(
        x(t) = \sum_{k=-\infty}^\infty c_k e^{-ik\oz t}
    \) vilket är Fourierserien på komplex form. \(
        c_k
    \) anger amplitud hos varje komplex exponential i formeln ovan.
\end{påm}

\(
    c_k 
\) och \(
    X(\io)
\) ger information om en signals frekvensinnehåll. Signalens amplitud fördelad
över de olika frekvenserna är \(
    \abs{c_k}
\) eller \(
    \abs{X(\io)}
\) i det periodiska och icke-periodiska fallet. Vi kan från detta skapa 
amplitudspektrum. Signalens fas fördelad över de olika frekvenserna är \(
    \arg{c_k}
\) och \(
    \arg{X(\io)}
\) för de två huvudfallen.

Från Parsevals formel fås för en periodisk signal (effektsignal) den totala
medeleffekten \(
    \overline{P} = \inv{T} \int_T \abs{x(t)}^2 \dd t 
    = c_0^2 + \sum_{k=1}^\infty 2 \abs{c_k}^2
\) vilket kan kallas effekttäthetsspektrat. Den kan delas upp efter frekvens 
om man vill. För en kontinuerlig signal (energisignal) definierar vi
den totala energin \(
    E = \infint \abs{x(t)}^2 \dd t 
    = \inv{2\pi} \infint \abs{X(\io)}^2 \dd \omega
\). Det är ett energi(täthets)spektrum och kan delas upp per frekvens.

\subsection{Systemanalys}
\(
    y(t) = \infint h(\tau) x(t-\tau) \dd \tau
\). Låt \(
    x(t) = e^{\io t}
\). Då är \(
    y(t) = h(t) * x(t)
    = \infint h(\tau) e^{\io (t-\tau)} \dd \tau
    = \infint h(\tau) e^{\io t} e^{-\io \tau} \dd \tau
    = e^{\io t} \infint h(\tau) e^{-\io \tau} \dd \tau
    = e^{\io t} H(\io)
\). 

I vår Fourierserie ingår frekvenserna \(
    k\oz, k \in \intn
\). 

\(
    x_k(t) = e^{ik\oz t} \to y_k(t) = e^{ik\oz t} H(ik\oz)
\) 

\(
    x_{-k}(t) = e^{-ik\oz t} \to y_{-k}(t) = e^{-ik\oz t} H(-ik\oz)
\) 

\(
    x(t) = \half (x_k(t) + x_{-k} (t)) = \cos(k\oz t) 
    \to y(t) = \half y_k(t) + \half y_{-k}(t)
\). Låt \(
    H(ik\oz) = H_k = \abs{H_k} e^{i\theta_k}\in \compn
\). Då är \(
    H(-ik\oz) = H_{-k} = \abs{H_k} e^{-i\theta_k}\in \compn
\). Om man summerar (använder superposition) får man då \(
    y(t) = \half \abs{H_k} \left( e^{ik\oz t} e^{i\theta_k} + e^{-ik\oz t} e^{i\theta_k} \right)
    = \abs{H_k} \cos(k\oz t + \theta_k)
\) vilket är den ursprungliga signalen med en amplitudpåverkan och en 
faspåverkan. 

\(
    H_k = H(ik\oz) 
\) är systemets \emph{frekvenssvar} (\(
    H(\io)
\)). Den ändrar amplitud och fas på varje sinusformad med frekvensen \(
    \omega
\).

\end{document}