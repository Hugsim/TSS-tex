\documentclass[a4paper]{article}
\usepackage{../flpack}

\begin{document}

\providecommand\fname{}
\renewcommand\fname{19-09-12}

\subsection{Approximation av signaler}
Låt \(
    f
\) vara en signal och \(
    x
\) vara en modellsignal som vi förstår. Vi vill modellera \(
    f
\) m.h.a.\ \(
    x
\), så ungefär \(
    f \approx cx \implies \text{ Ett fel } e = f - cx
\).

Kom ihåg linalgens \enquote{inre produktrum}:

\(
    V
\) är ett vektorrum över \(
    \realn
\)  eller \(
    \compn
\), \(
    \lang \cdot, \cdot \rang 
\), en inre/skalär produkt på \(
    V
\). Den uppfyller \(
    \lang x, y \rang = \lang y, x\rang 
\), \(
    \lang x, x \rang \geq 0 (= 0 \iff x = 0)
\) och \(
    \lang x, \lambda y_1+y_2 \rang = \lang x, y_1 \rang \lambda + \lang x, y_2 \rang
\).

\(
    f_1, f_2
\) är signaler, då är här \(
    \lang f_1, f_2 \rang = \bi_{-\infty}^\infty \bar{f_1(t)} f_2(t) \dd t
\).

I det periodiska fallet är \(
    \lang f_1, f_2 \rang = \bi_0^T \bar{f_1(t)} f_2(t) \dd t
\).

Mellan två signaler \(
    f_1, f_2
\) har vi en vinkel \(
    \theta
\) som definieras genom \(
    \cos(\theta) = \bif{\lang f_1, f_2 \rang}{\abs{f_1} \cdot \abs{f_2}} 
\) där \(
    \abs{f} = \sqrt{\lang f, f \rang}
\).

Felet \(
    e
\) är litet om \(
    \abs{e}
\) är litet. När är då \(
    \abs{e}
\) som minst? D.v.s.\ vad är \(
    \min{\abs{f-cx}}
\) över \(
    \compn
\)? Går att härleda att det uppnås för \(
    c = \bif{\lang x, f \rang}{\abs{x}^2} 
\). Det betyder att \(
    f \approx \bif{\lang x, f \rang}{\abs{x}^2} x
\).

\dquote{Jag växte också upp på Lindholmen.}{Mattesnubben}

Om \(
    f
\) är en signal definierad för \(
    -\infty < t < \infty
\) är dess energi \(
    E_f = \bi_{-\infty}^{\infty} \abs{f(t)}^2 \dd t = \abs{f}^2
\).

Om \(
    f_1, f_2
\) är signaler definierade för \(
    -\infty < t < \infty
\) är \(
    \bif{
        \bi_{-\infty}{\infty} f_1(t) f_2(t) \dd t
    }{\sqrt{E_{f_1}E_{f_2}}} = \bif{\lang f_1, f_2 \rang}{\abs{f_2}\abs{f_2}}  
\) korrelationen mellan signalerna. Den är \(
    1 \iff f_1 = f_2
\) och \(
    -1 \iff f_1 = -f_2
\).

Anta att vi har flera modellsignaler \(
    x_1, x_2, \dots, x_N
\) så att \(
    f \approx c_1x_1 + c_2x_2 + \cdots + c_N x_N
\). Vi vill då minimera \(
    e = f - \bs_{n=1}^{N}c_k e_k
\). Det uppnås för \(
    c_k = \bif{\lang x_n, f\rang }{\abs{x_n}^2} 
\) \textbf{OM} \(
    x_1, x_2, \dots, x_n
\) inte är korrelerade, d.v.s.\ de är ortogonala eller den inre/skalärprodukten är \(
    0
\).

\subsection{Fourierserier}
Fourierserier appliceras på periodiska signaler. Låt \(
    f 
\) vara en signal med period \(
    T
\). 

Vi tittar på följande modellsignaler: \(
    \{1, \cos(\omega_0 t ), \dots, \cos(N \omega_0 t), \sin(\omega_0 t ), \dots, \sin(N \omega_0 t)\}
\).

\(
    \lang \cos(k \omega_0t), \cos(l \omega_0 t)\rang = \bi_0^T \cos(k \omega_0t) \cos(l \omega_0 t) \dd t 
    = \bi_0^T \bif{e^{jk\omega_0t}+e^{-jk\omega_0t}}{2}  \bif{e^{jl\omega_0t}+e^{-jl\omega_0t}}{2} \dd t
    = \bif{1}{4} \bi_0^T e^{j(k+l)\omega_0t} + e^{j(k-l)\omega_0t} +  e^{j(-k+l)\omega_0t} +  e^{-j(k+l)\omega_0t} \dd t
    = T \text{ om } k = l = 0, \bif{T}{2} \text{ om } k = l > 0, 0 \text{ annars.} 
\) På samma sätt är \(
    \bi_0^T \cos(k \omega_0t) \sin(l \omega_0 t) \dd t = 0
\) och \(
    \bi_0^T \sin(k \omega_0t) \sin(l \omega_0 t) \dd t = \bif{T}{2} \text{ om } k = l \text{ och } 0 \text{ annars.} 
\) 

Alltså är den bästa approximationen \[
    f \approx \bif{a_0}{2} + \bs_{k=1}^{N} a_k \cos(k\omega_0t)+ b_k \sin(k\omega_0t)
\] där \(
    a_k = \bif{2}{T} \bi_0^T f(t) \cos(k \omega_0 t)\dd t = \bif{\lang \cos(k \omega_0 t), f \rang}{\sqabs{\cos(k \omega_0t)}} 
\) och \(
    b_k = \bif{2}{T} \bi_0^T f(t) \sin(k \omega_0 t)\dd t = \bif{\lang \sin(k \omega_0 t), f \rang}{\sqabs{\sin(k \omega_0t)}}
\).

\begin{sats}
    Om \(
        f
    \) är \(
        T
    \)-periodisk och \(
        \bi_0^T \abs{f(t)} \dd t < \infty 
    \) existerar \(
        a_k
    \) och \(
        b_k
    \).
\end{sats}

\begin{sats}
    Om \(
        f
    \) dessutom är kontinuerlig på intervallet på \(
        0 \leq t \leq T
    \) utom i ändligt många punkter och \(
        f(t+) = \blim_{h \to 0} f(t+h)
    \) och \(
        f(t-) = \blim_{h \to 0} f(t-h)
    \) existerar för alla \(
        t
    \) så är \(
        \bif{f(t+)+f(t-)}{2} = \blim_{N\to\infty} \bif{a_0}{2} + 
        \bs_{k=1}^N a_k \cos(k \omega_0 t) + b_n \sin(k \omega_0 t)
    \).
\end{sats}

\dquote{Det här är högst icke-trivialt.}{Mattesnubben}

\dquote{Om man stoppar in något i datorn som itne konvergerar kommer det gå åt röven.}{Mattesnubben}

\begin{ex}
    \f

    Vi har ju \(
        f = \bif{a_0}{2} + \bs_{k=1}^N a_k \cos(k \omega_0 t) + b_n \sin(k \omega_0 t)
    \), för \(k \geq 0 \text{ har vi } 
        a_k = \bif{2}{T} \bi_0^T f(t) \cos(k \omega_0 t)\dd t = \bif{2}{T} \bi_0^{\frac{T}{2}} \cos(k \omega_0 t)\dd t = 
        1 \text{ för } k = 0, 
        \bif{2}{Tk\omega_0}[\sin(k\omega_0t)]_0^{\bif{T}{2}} 
        = 1 \text{ för } k = 0, 0 \text{ annars.} 
    \) och för \( k > 0 \text{ har vi } 
        b_k = \bif{2}{T} \bi_0^T f(t) \sin(k \omega_0 t)\dd t = 
        \bif{2}{T} \bi_0^{\frac{T}{2}} \sin(k \omega_0 t)\dd t =
        \bif{2}{Tk\omega_0}[-\cos(k\omega_0t)]_0^{\frac{T}{2}} =
        \bif{1}{\pi k} (-\cos(k\pi) + 1) =
        \bif{1}{\pi k} (- (-1)^k + 1) =
        0 \text{ för } k = 2m \text{ och } \bif{2}{\pi(2m+1)} \text{ annars.}  
    \)

    \(
        f = \half + \bif{2}{\pi} \bs_{m=0}^\infty \bif{\sin((2m+1)\omega_0 t)}{\pi(2m+1)} 
    \) 
\end{ex}

\subsubsection{Kompakt form}
\begin{align*}
    f &= \bif{a_0}{2} + \bs_{k=1}^\infty a_k \cos(k \oz t) + b_k \sin(k \oz t) \\
      &= \bs_{n=0}^\infty c_n \cos(k \oz t + \theta_n)
\end{align*}

\(
    c_n \cos(k \oz t + \theta_n) = c_n \cos(\theta_n) \cos(k \oz t) -  c_n \sin(\theta_n) \sin(k \oz t)
\)    

\(
    \tan(\theta_n) = \bif{-b_n}{a_n} 
\) och \(
    c_n^2 = a_n^2 + b_n^2
\) 

\subsection{Frekvensspektrum}
\f

Används för att visualisera fourierserier.

\begin{ex}
    \f

    \(
        f = \bif{a_0}{2} + \bs_{k=1}^N a_k \cos(k \omega_0 t) + b_n \sin(k \omega_0 t)
    \), och vi börjar med fallet där \(
        k > 0
    \). Då är \(
        a_n = \bif{2}{T} \bi_0^{\frac{T}{2}} t \cos(k \oz t) \dd t
    \) och med partiell integrering får vi \(
        a_n = \bif{2}{T} \left[t \bif{\sin(k \oz t)}{k\oz }  \right]_0^{\frac{T}{2}} - \bif{2}{T}  \bi_0^{\frac{T}{2}} \bif{\sin(k \oz t)}{k \oz} \dd t
        = 0 + \bif{2}{T\oz^2k^2}\left[\cos(k\oz t)\right]_0^{\frac{T}{2}} 
        = 0 \text{ när } k = 2m, \bif{-2}{\pi \oz(2m+1)^2} \text{ annars.}  
    \) Vi har också \(
        a_0 = \bif{2}{T} \bi_0^T f(t) \dd t = \bif{2}{T} \bi_0^{\frac{T}{2}} t \dd t = \bif{T}{2} 
    \).

    Om man räknar på \(
        b_n
    \) blir det likt, men till slut får vi \(
        f = \bif{T}{4} + \bif{2}{\pi} \bs_{m=0}^\infty \bif{-\cos((2m+1)\oz t)}{\oz(2m+1)^2} + \bif{\sin((2m+1)\oz t)}{2m+1}
    \) eller i kompakt form \(
        \bs_{n=0}^\infty c_n \cos((2n+1) \oz t +\theta_n)
    \) där \(
        \theta_n = \arctan(\bif{-b_{2n+1}}{a_{2n+1}}) = \arctan(\oz(2n+1))
    \).
\end{ex}

\end{document}