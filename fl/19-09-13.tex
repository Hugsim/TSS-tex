\documentclass[a4paper]{article}
\usepackage{../flpack}

\begin{document}

\providecommand\fname{}
\renewcommand\fname{19-09-13}

\dquote{Det är ju helg snart, man får roa sig med något}{Mattesnubben om ingenstans deriverbara funktioner}
\dquote{Då övergår vi till saltgruvorna då.}{Mattesnubben}
\subsection{Exponentiella Fourierserier}
Låt \(
    f
\) vara en \(
    T
\)-periodisk signal. 

Vi kan skriva \(
    f = \bs_{k = -\infty}^\infty c_k(f) e^{i k \oz t}
\) eftersom 
\begin{align*}
    f &= \bif{a_0}{2} + \bs_{k=1}^\infty a_k \cos(k \oz t) + b_k \sin(k \oz t) \\
      &= \bif{a_0}{2} \bs_{k=1}^\infty \left( \bif{a_k}{2} + \bif{b_k}{2i}  \right) e^{j k \oz t} + \left( \bif{a_k}{2} - \bif{b_k}{2i}  \right) e^{-j k \oz t} \\
      &= c_0 \bs_{k=1}^\infty c_ke^{ik\oz t} + \bs_{k=-\infty} ^{-1} c_k e^{ik\oz t} \\
      &= \bs_{k = -\infty}^\infty c_k(f) e^{i k \oz t}
\end{align*}
där \(
    c_k = \bif{a_k}{2}  + \bif{b_k}{2i} = \dots \text{(se andras anteckningar)} 
    = \inv{T} \bi_0^T f(t) e^{-jk\oz t} \dd t
\).

\begin{anm}
    Går lätt att visa med definitionen av faltning att om en av insignalerna är periodiska är också utsignalen det.
\end{anm}

\begin{defn}[Laplace-/Fouriertransform]
    \[
        H(s) = \bi_{-\infty}^\infty h(u) e^{-su} \dd u
    \] 
    givet att \(
        h
    \) är stabil.
\end{defn}

Om \(
    x = \bs_{k = -\infty}^\infty c_k e^{jk\oz t}
\) är \(
    h * x = \bs_{k = -\infty}^\infty c_k h * e^{jk\oz t}
    = \bs_{k = -\infty}^\infty c_k H(jk\oz) e^{jk\oz t}
\).

\subsection{Faltningsekvationer (periodiska fallet)}
Vi vill lösa \(
    y = h * x
\).

Anta att \(
    y, h
\) är kända och att \(
    y
\) är periodisk. Då kan vi skriva \(
    y = \bs_{k = -\infty}^\infty c_k(y) e^{jk\oz t}
\).

Vi vet då att \(
    \bs_{k=-\infty}^\infty c_k(y) e^{jk\oz t} 
    = \bs_{k=-\infty}^\infty c_k(x) H(jk\oz)e^{jk\oz t}
\) vilket ger att \(
    \forall k : c_k(y) = c_k(x) H(jk\oz) \iff \forall k : c_k(x) = c_k(y) \cdot \inv{H(jk\oz)}
\)  

\subsection{Faltning på kompakt form}
\[
    H = \abs{H}e^{i \angle H}
\] 

Om vi då har signalen \(
    x = \bs c_k(x) \cos(k\oz t + \theta_k(x))
\) är \(
    h * x(t) = \bs_{k=0}^\infty c_k h * \left( \bif{e^{j(k\oz t + \theta_k(x))} + e^{-j(k\oz t + \theta_k(x))}}{2}  \right)
    = \bs_{k=0}^\infty c_k\left( H(jk\oz) \bif{e^{j(k\oz t + \theta_k(x))}}{2} + H(-jk\oz) \bif{e^{-j(k\oz t + \theta_k(x))}}{2}  \right)
\) 

\begin{påst}
    Om \(
        h
    \) är reell är \(
        H(\overline{s}) = \overline{H(s)}
    \).

    \begin{proof}
        \begin{align*}
            H(\overline{s}) &= \bi_{-\infty}^\infty h(u) e^{-\overline{s}u} \dd u \\
            &= \overline{ \bi_{-\infty}^\infty h(u) e^{-su} \dd u } \\
            &= \overline(H(s))
        \end{align*}
    \end{proof}
\end{påst}
\begin{align*}
    &\bs_{k=0}^\infty c_k\left( H(jk\oz) \bif{e^{j(k\oz t + \theta_k(x))}}{2} + H(-jk\oz) \bif{e^{-j(k\oz t + \theta_k(x))}}{2}  \right) \\
    = &\bs_{k=0}^\infty c_k\left( H(jk\oz) \bif{e^{j(k\oz t + \theta_k(x))}}{2} + \overline{H(jk\oz)} \bif{e^{-j(k\oz t + \theta_k(x))}}{2}  \right) \\
    = &\bs_{k=0}^\infty c_k \bif{\abs{H(jk\oz)}}{2} \left( e^{j\angle H(jk\oz)} e^{j(k\oz t + \theta_k)} + e^{-j\angle H(jk\oz)} e^{-j(k\oz t + \theta_k)} \right)  \\
    = &\bs_{k=0}^\infty c_k \abs{H(jk\oz)} \cos(k\oz t + \theta_k + \angle H(jk\oz))
\end{align*}

\subsection{Derivata}
Låt \(
    x(t) = \bif{a_0}{2} + \bs_{k=1}^\infty a_k(x) \cos(k \oz t) + b_k(x) \sin(k \oz t)
\). Då är \(
    x'(t) = \bs_{k=1}^\infty a_k(x)(-k\oz)\sin(k \oz t) + b_k(x) (k\oz)\cos(k \oz t)
\). \(
    a_0(x') = 0, a_k(x') = k\oz b_k(x), b_k(x') = -k\oz a_k(x)
\) för \(
    k > 0
\).

\begin{ex}
    Låt \(
        h = ue^{at}
    \) (enhetssteg), d.v.s.\ \(
        h(t) = e^{at} 
    \) för \(
        t \geq 0
    \) och \(
        0
    \) annars.

    Låt också \(
        x(t) = \cos(t) + \sin(2t) + \cos(3t - \frac{\pi}{3})
        = \half e^{it} + \half e^{-it} + \bif{e^{it}}{2i} - \bif{e^{-it}}{2i}
        + \half e^{3it} e^{-\frac{\pi}{3}} + \half e^{-3it} e^{\frac{\pi}{3}}
        = c_{-3} e^{-3it} + c_{-2} e^{-2it} + c_{-1} e^{-it} + c_1 e^{it} + c_2 e^{it} + c_3 e^{3it}
    \). Då är \(
        c_{-3} = \bif{e^\frac{\pi i}{3} }{2}, 
        c_{-2} = -\bif{1}{2i}
        c_{-1} = \half , 
        c_1 = \half,
        c_2 = \bif{1}{2i}
    \) och \(
        c_3 = \bif{e^{-\frac{\pi i}{3}} }{2}
    \).

    Då är \(
        h * x(t) =  \bs_{k=-\infty}^\infty c_k H(jk\oz) e^{ik\oz t}
    \). \(
        H(s) = \bi_0^\infty h(t) e^{-st} \dd t
        = \bi_0^\infty e^{(a-s)t} \dd t
        = \inv{s-a} \text{ när } \Re{s} > \Re{a} \text{ och odef.\ annars.}
    \) \(
        H 
    \) är definierad, givet att \(
        s = ik\oz
    \) när \(
        \Re{a} < 0
    \), d.v.s.\ att \(
        h
    \) är stabil.

    \(
        H(jk\oz) = \bif{1}{ik\oz -a} 
    \) vilket säger att \(
        h * x(t) = \bs_{k=0}^\infty \bif{c_k}{ik\oz - a} e^{ik\oz t}
    \). 

    Med \(
        x
    \) som ovan blir \(
        h*x(t) = \bs_{k=-3}^3 \bif{c_k}{ik\oz -a} e^{ik\oz t} 
        = \bif{e^{\frac{\pi i}{3}}}{2} \inv{-3i\oz-a}e^{-3it} 
        + \bif{1}{2i} \inv{-2i\oz-a}e^{-2it} + \cdots
    \).

    \(
        D(h*x) = \bs c_k \cdot \bif{ik\oz }{ik\oz - a} e^{ik\oz t}
        = \bs c_k e^{ik\oz t} + \bs c_k \cdot \bif{a}{ik\oz - a} e^{ik\oz t} 
        = x + a (h * x)
    \), d.v.s.\ \(
        y=h*x 
    \) löser \(
        (D-a)y = x
    \).
\end{ex}

\dquote{Riktigt jävla konkret.}{Mattesnubben}

\subsection{Parsevals sats}
Satsen säger saker om \(
    \bi_0^T \abssq{x(t)} \dd t
\). 

\begin{påm}
    \(
        P_x = \blim_{R \to \infty} \inv{2R} \bi_{-R}^R \abssq{x(t)} \dd t
        = \blim_{m \to \infty} \inv{2mT} \bi_{-mT}^{mT} \abssq{x(t)} \dd t
        = \blim_{m \to \infty} \inv{2mT} \left( 2m \bi_{0}^{T} \abssq{x(t)} \dd t \right)
        = \bif{1}{T} \bi_0^T \abssq{x(t)} \dd t
    \)
\end{påm}

\(
    x = \bs_{k=\infty}^\infty c_ke^{ik\oz t} \implies \bi_0^T \abssq{x(t)} \dd t 
    = \abssq{x} = \lang x, x, \rang 
    = \lang \bs_{k=-\infty}^\infty c_ke^{ik\oz t}, \bs_{l=-\infty}^\infty c_le^{il\oz t}\rang
    = \bs_{k, l} \overline{c_k} c_l \lang e^{ik\oz t}, e^{il\oz t}\rang
    = T \bs_{k = -\infty}^\infty \abssq{c_k} 
    = \bi_0^T e^{-ik\oz t}e^{il\oz t} \dd t
\) vilket är \(
    0
\) för \(
    k\neq l
\) och \(
    1 \text{ annars.} 
\) Sammanfattningsvis är satsen \[
    P_x = \bs_{k=-\infty}^\infty \abssq{c_k} 
    = \bif{\abssq{a_o}}{4} + \half \bs_{k=1}^\infty \abssq{a_n} + \abssq{b_n} 
    = \abssq{c_0} + \half\bs_{k=1}^\infty \abssq{c_k} 
\]

\end{document}