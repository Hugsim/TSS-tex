\documentclass[a4paper]{article}
\usepackage{../flpack}

\begin{document}

\providecommand\fname{}
\renewcommand\fname{19-10-16}

\subsection{Diskreta LTI-system}
De kan beskrivas med differensekvationer.
\(
    y[n] + \sum_{k=1}^N a_k y[n-k] = \sum_{k=0}^M b_k x[n-k]
\) vilket om man z-transformerar ger \(
    Y(z) + \sum_{k=1}^N a_k z^{-k} Y(z) = \sum_{k=0}^M b_k z^{-k} X(z)
        = Y(z) (1 + \sum_{k=1}^N a_k z^{-k}) = X(z) \sum_{k=0}^M b_k z^{-k}
\) vilket vi kallar \(
    Y(z) \cdot A(z) = X(z) \cdot B(z)
\). Det ger \(
    Y(z) = \frac{\sum_{k=0}^M b_k z^{-k}}{1 + \sum_{k=1}^N a_k z^{-k}} \cdot X(z)
\) där vi kallar kvoten för \(
    H(z) = \frac{A(z)}{B(z)} 
\).

\(
    H(z) = \frac{B(z)}{A(z)} 
\) är systemets överföringsfunktion vilket är en kvot mellan två polynom i \(
    z^{-1}
\). 

En vanlig form hos z-transformen i våra tillämpningar är som en kvot mellan 
polynom i \(
    z^{-1}
\) eller \(
    z
\), d.v.s.\ \(
    H(z) = \frac{b_0 + b_1z^{-1} + \cdots + b_M z^{-M}}{a_0 + a_1 z^{-1} + \cdots + a_N z^{-N}} 
\). 

Om vi faktoriserar får vi \(
    H(z) = b_0 \cdot \frac{\prod_{k=1}^M (1-c_k z^{-1})}{\prod_{k=1}^N (1-d_k z^{-1})} 
\). \(
    z = c_k
\) ger nollställen (\(
    \circ
\)) och \(
    z=d_k
\) ger poler (\(
    \times
\)). Beskrivs grafiskt genom att rita dem med sina tecken i z-planet.



\end{document}