\documentclass[a4paper]{article}
\usepackage{../flpack}

\begin{document}

\providecommand\fname{}
\renewcommand\fname{19-09-25}

\dquote{När ni lyssnar på Spotify är det ju inte ettor och nollor som strömmar in i örat på er.}{Antman}
\subsection{Rekonstruktion}
\begin{sats}[Samplingsteoremet]
    Låt \(
        x(t)
    \) ha Fouriertransformen \(
        X(\io)
    \) och vara en bandbegränsad signal, d.v.s.\ \(
        X(\io) = 0 
    \) för något \(
        \abs{\omega} > \omega_M
    \).
    
    Då, om samplingsfrekvensen \(
        \omega_s > 2\omega_M
    \) kan \(
        x(t)
    \) återskapas utifrån sina sampelvärden \(
        x[n] = x(nT), n \in \intn
    \) där \(
        \omega_s = \frac{2\pi}{T} 
    \).
\end{sats}

\begin{defn}[Nyquistfrekvens]
    Halva samplingsfrekvensen \(
        \frac{\omega_s}{2}
    \) kallas \emph{Nyquistfrekvensen}.
    Den jämförs med \(
        \omega_M
    \).
\end{defn}

Om \(
    x(t)
\) inte är bandbegränsad kan man använda ett 
\emph{antivikningsfilter/anti aliasing filter}, d.v.s.\ ett kontinuerligt 
lågpassfilter. Det reducerar bandbredden hos en signal \(
    x(t)
\) och appliceras innan sampling. Tanken är att det ska ta bort 
frekvenser som är större än \(
    \frac{\omega_s}{2} 
\) och därmed minska effekten av vikning/aliasing.

\subsubsection{Ideal rekonstruktion}
Handlar om att återskapa \(
    x(t)
\) utifrån sina sampelvärden \(
    x[n] \coloneqq x(nT)
\). 

\(
    x(t) \xleftrightarrow{\text{FT}} X(\io)
\) och \(
    x_p(t) \xleftrightarrow{\text{FT}} X_p(\io) 
        = \inv{T} \sum_{k=-\infty}^\infty X(i(\omega-k\omega_s))
\). Applicera ett idealt rekontruktionsfilter/-system 
\[
    H_r(\io) = 
    \left\{\begin{matrix}
        T & \abs{\omega} < \frac{\omega_s}{2}\\ 
        0 & \text{annars} 
    \end{matrix}\right.
\] 

Vi får då \(
    Y(\io) = H_r(\io)X_p(\io)
\) och i tidsdomänen \(
    y(t) = h_r(t) * x_p(t) 
         = h_r(t) * (\sum_{n=-\infty}^\infty x[n] \delta(t-nT))
         = \sum_{n=-\infty}^\infty x[n] h_r(t-nT)
\). \(
    y(t)
\) är då en summa av viktade och skiftade impulssvar \(
    h_r(t)
\). Nu vill vi beräkna de impulssvaren. 
\begin{align*}
    h_r(t) &= \inv{2\pi} \infint H_r(\io) e^{\io t} \dd \omega \\
           &= \frac{T}{2\pi} \int_{-\frac{\omega_2}{2}}^{\frac{\omega_s}{2}} 
                e^{\io t} \dd \omega \\
           &= \frac{T}{2\pi} \left[ \frac{e^{\io t}}{it} \right]_{-\frac{\omega_2}{2}}^{\frac{\omega_s}{2}} \\
           &= \frac{T}{\pi t} \inv{2i} \left(e^{i \frac{\omega_s}{2}t} - e^{-i \frac{\omega_s}{2}t}\right) \\
           &= \frac{T}{\pi t} \sin(\frac{\omega_s}{2} t)  \\
           &= \{T = \frac{2\pi}{\omega_2} \} \\
           &= \frac{2\pi}{\omega_s \pi t} \sin(\frac{\omega_s}{2} t)  \\
           &= \frac{\sin(\frac{\omega_s}{2} t)}{\frac{\omega_s}{2} t} \\
           &= \text{sinc}\left(\frac{\omega_s}{2} t\right). \\
\end{align*}

\begin{defn}[sinc]
    \(
        \text{sinc}(x) \coloneqq \frac{\sin(x)}{x} 
    \) 

    \f
\end{defn}

Ett praktiskt problem med detta är att \(
    h_r(t)
\) inte är kausalt och har oändlig utsträckning.

\subsubsection{Praktisk rekonstruktion}
För att göra det praktiskt att rekonstruera måste systemet vara kausalt.
Vi gör det genom att arbeta med hållkretsar, och vi börjar med nollte 
ordningens hållkrets. Det är ett system där \(
    x_p(t)
\) ger en utsignal \(
    x_0(t)
\) med impulssvaret
\[
    h_0(t) = \left\{\begin{matrix}
                 1 & 0 \leq t < T\\ 
                 0 & \text{annars} 
             \end{matrix}\right.
\] 

\f

Sampelvärdet hålls kvar på sin nivå tills nästa sampelvärde kommer.

\(
    x_0(t) = h_0(t) * x_p(t)
           = \{x[n] = x(nT)\}
           = h_0(t) * \sum_{n=-\infty}^\infty x[n] \delta(t-nT)
           = \sum_{n=-\infty}^\infty x[n] h_0(t-nT)
\). Nu analyserar vi med Fouriertransformen/i frekvensdomänen: \(
    X_0(\io) = H_0(\io) X_p(\io)
\). Då är \(
    h_0(t) \xleftrightarrow{\text{FT}} 2e^{-i \frac{\omega T}{2} } \frac{\sin(\frac{\omega T}{2} )}{\omega} 
\) vilket i princip är en \(
    \text{sinc} 
\)-funktion.

\subsubsection{Aliasing/vikning}
\begin{ex}
    Låt \(
        x(t) = A \cos(\omega_M t) = \frac{A}{2} (e^{\io_M t} + e^{-\io_M t}
    \).

    Fouriertransformen av det är \(
        X(\io) = A \pi (\delta(\omega - \omega_M) + \delta(\omega+\omega_M))
    \).

    \f

    Sen samplar vi vilket ger \(
        X_p(\io) = \inv{T} \sum_{k=-\infty}^\infty X(i(\omega - k \omega_s))
    \). Om \(
        \omega_M < \frac{\omega_s}{2} 
    \) får vi 

    \f

    Med ett idealt rekonstruktionsfilter \(
        H_r(\io)
    \) får vi \(
        H_r(\io) X_p(\io) 
            = T \frac{A \pi}{T} (\delta(\omega-\omega_M) + \delta(\omega+\omega_M))
    \). Vi får alltså tillbaka \(
        x(t) = A \cos(\omega_M t)
    \) 

    Annars, om \(
        \omega_M > \frac{\omega_s}{2}
    \) får vi  

    \f

    Med det ideala rekonstruktionsfiltret får vi då \(
        X_p(\io) H_r(\io) 
            = A \pi (\delta(\omega-(\omega_s-\omega_M)) + \delta(\omega+(\omega_s - \omega_M)))
    \) vilket svarar mot \(
        x_1(t) = A \cos((\omega_s-\omega_M) t)
    \). 

    Med ett numeriskt exempel med konstanterna 
    \begin{align*}
        \omega_s &= \SI{200}{\radian\per\second} \\
        \omega_M &= \SI{120}{\radian\per\second} 
    \end{align*}

    Den rekonstruerade signalen har då vinkelfrekvensen \(
        \omega_s-\omega_M = 200-120= \SI{80}{\radian\per\second}
    \). Det är en \enquote{falsk} frekvens, d.v.s.\ aliasing/vikning.
\end{ex}

\end{document}