\documentclass[a4paper]{article}
\usepackage{../flpack}

\begin{document}

\providecommand\fname{}
\renewcommand\fname{19-10-09}

\subsection{Sammanfattning}
\begin{table}[]
    \begin{tabular}{llllllllll}
        \(X(\io)\)     & \(\xleftrightarrow{\text{FT}}\) & \(x(t)\)       & \(\xrightarrow{\text{Sampling}}\) & \(X[n]\)       & \(\xleftrightarrow{\text{DTFT}}\) & \(X(e^{\iO})\) & \(\xdashrightarrow{\Omega = \Omega_k = \frac{2\pi}{N}k}\) & \(X[k]\)  &  \\
        Kont.   &                             & Kont.   &                               & Diskret        &                               & Kont.  &                                                       & Diskret   &  \\
        Icke-per. &                             & Icke-per. &                               & Icke-per. &                               & Per.     &                                                       & Per. &  \\
                    &                             &                &                               &                &                               &               &                                                       & DFT/FFT   & 
    \end{tabular}
\end{table}

Den kontinuerliga signalens Fouriertransform kan erhållas ur den samplade 
signalens Fouriertransform med \(
    \abs{\Omega} < \pi
\) om effekten av aliasing kan göras tillräckligt liten. 

\subsection{Mer om DFT: Samband mellan \(
    k
\), \(
    \omega
\) och \(
    \Omega
\)}

\(
    x(t) = \sin(\omega t)
\) samplas vid \(
    t = nT \implies x[n] = \sin(\omega T n) = \sin(\Omega n)
\). 

\subsubsection{DFT:s frekvensaxel}
Kan vara mot \(
    \Omega = \frac{2\pi}{N} k
\), \(
    k
\), \(
    \omega = \text{(Om } \Omega = 2\pi \text{)} \implies \omega = \frac{2\pi}{T} = \omega_s
\) eller \(
    f_s = \frac{\omega_s}{2\pi} 
\) i \(
    \si{\hertz} 
\), beror lite på.

\f

Notera uppbyggnaden av syntesekvationen för DFT som ger \(
    x[n]
\) utifrån \(
    X[k]
\). Den är ju \(
    x[n] = \inv[N] \sum_{k=0}^{N-1} X[k] e^{i \frac{2\pi}{N} k n}
\), vilket är en superposition av basfunktionen \(
    e^{i \frac{2\pi}{N} k n}
\) vilket vi kallar för \(
    \phi_k[n]
\) med viktfunktionen \(
    X[k]
\). Några egenskaper för \(
    \phi_k[n]
\), där \(
    k,n\in \intn
\), är att 
\begin{itemize}
    \item \(
        \phi_k[n+N] = e^{i \frac{2\pi}{N} k (n+N)} 
            = e^{j \frac{2\pi}{N} kn} \cdot e^{j \frac{2\pi}{N} kN}
            = \phi_k[n] \cdot 1
            = \phi_k[n]
    \), alltså är den periodisk i \(
        n
    \) med perioden \(
        N
    \).

    \item \(
        \phi_{k+N}[n] = e^{i \frac{2\pi}{N} (k+N)n} 
        = e^{j \frac{2\pi}{N} kn} \cdot e^{j \frac{2\pi}{N} nN}
        = \phi_k[n] \cdot 1
        = \phi_k[n]
    \), alltså är den också periodisk i \(
        k
    \)  med perioden \(
        N
    \). 
\end{itemize}

Det finns alltså bara \(
    N
\) stycken unika bassignaler, d.v.s.\ \(
    \phi_0[n] = \phi_N[n]
\), \(
    \phi_1[n] = \phi_{N+1}[n]
\) o.s.v.

Om det är ett heltal antal perioder i samplingen ger det exakt vilket \(
    k
\) som ensamt bidrar. Både \(
    k
\) och \(
    -k = N - k
\) finns med. 

\dquote{Det ser regelbundet ut, vilket jag är tacksam för.}{Ants om sin EKG}

\subsection{Introduktion till \(
    z
\)-transformen}

Låt \(
    z
\) vara en komplex variabel, alltså är \(
    z = a + bi = r e^{\iO}
\). Kom också ihåg att \(
    z^n = (re^\iO)^n 
        = r^n e^{\iO \cdot n} 
        = r^n(\cos(\Omega n) + i \sin(\Omega n))
\). Låt \(
    z^n
\) utgöra insignal till ett diskret LTI-system med impulssvar \(
    h[n]
\). Faltningssumman ger att \(
    y[n] = h[n] * x[n] 
        = \sum_{k=-\infty}^{\infty} h[k] \cdot x[n-k]
        = \sum_{k=-\infty}^\infty h[k] \cdot z^{n-k} 
        = \sum_{k=-\infty}^\infty h[k] \cdot z^n z^{-k}
        = z^n \sum_{k=-\infty}^\infty h[k] \cdot z^{-k}
        \coloneqq z^n \cdot H(z) = \mathcal{z}\{h[n]\} % Konstigt z?
\) där \(
    H(z)
\) är \(
    z
\)-transformen av \(
    h[n]
\). 

\begin{defn}[\(
    z
\)-transformen]
    \(
        z
    \)-transformen av den diskreta signalen \(
        x[n]
    \) är \(
        X(z) = \sum_{k=-\infty}^\infty x[k] z^{-k}
    \).
\end{defn}

\end{document}