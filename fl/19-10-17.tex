\documentclass[a4paper]{article}
\usepackage{../flpack}

\begin{document}

\providecommand\fname{}
\renewcommand\fname{19-10-17}

\subsection{z-transformen för en generell sekvens}
Låt \(
    x[n]
\) vara en diskret signal. Då är \(
    X(z) = \sum_{n=-\infty}^\infty x[n]z^{-n}
\). Vi vet att vi kan skriva \(
    z = re^{\iO}
\). Då är \(
    X(z) = X(re^\iO) = \sum_{n=-\infty}^\infty x[n] {(re^{\iO})}^{-n}
        = \sum_{n=-\infty}^\infty (x[n] r^{-n}) e^{\iO n}
        = \text{DTFT} \{x[n] r^{-n}\}
\). Det konvergerar om \(
    \sum_{n=-\infty}^\infty \abs{x[n] r^{-n}} < \infty
\), d.v.s.\ om det är absolutsummerbart. Notera att konvergensen bara beror 
på \(
    \abs{z} = r
\) och \(
    x[n]
\). De värden på \(
    z
\) för vilka \(
    X(z)
\) konvergerar kallas för \enquote{konvergensområde} (ROC).

För en kausal signal gäller att om \(
    \sum_{n=0}^\infty \abs{x[n] r^{-n}} < \infty
\) för något \(
    r = r_0
\) måste den även konvergera för \(
    r > r_0
\). Konvergensområdet är alltså alla tal utanför en viss cirkel i det 
komplexa talplanet.

\subsubsection{Några egenskaper för konvergensområdet}
\begin{itemize}
    \item Inga poler får ligga i ROC
    \item Om DTFT existerar för signalen ligger enhetscirkeln i ROC (\(
              r=1
          \))
    \item Om \(
              x[n]
          \) är kausal och dess \(
              X(z)
          \) har poler svarar det minsta värdet på \(
              r
          \) för vilket \(
              X(z)
          \) konvergerar mot det största beloppet av \(
              X(z)
          \):s poler.
\end{itemize}

\subsection{Digitala filter}
Ett filter går att realisera med en differensekvation som beskrivs
av en kvot \(
    \frac{Y(z)}{X(z)} = H(z) 
        = \frac{b_0 + b_1 z^{-1} + \cdots + b_M z^{-M}}{1 + a_1z^{-1} + \cdots + a_N z^{-N}} 
\).

\subsubsection{Några slags filter}
\begin{itemize}
    \item FIR (Finite Impulse Response)
        \begin{itemize}
            \item Ändligt långt impulssvar (längd \(
                M < \infty
            \))
            \item \(
                a_1
            \), \(
                a_2
            \), \(
                \dots
            \), \(
                a_n = 0
            \) 
            \item Kan ges linjär fas
            \item Alltid stabila (\(
                M < \infty
            \), \(
                b_n < \infty
            \))
            \item Icke-rekursivt
        \end{itemize}
    \item IIR (Infinite Impulse Response)
        \begin{itemize}
            \item Oändligt långt impulssvar
            \item Rekursivt (Några \(
                a_k \neq 0
            \), \(
                k = 1,2,\dots,N
            \))
            \item Kan göras instabila
            \item För jämförbar amplitudkaraktäristik har lägre filterordning 
                      än ett FIR-filter (snabbare att räkna ut och modellera
                      eftersom det har färre multiplikationer att utföra)
        \end{itemize}
\end{itemize}

\subsection{Diskret system (filter)}
Låt \(
    H(z)
\) vara en överföringsfunktion till något system. Då är \(
    \eval{H(z)}_{z = e^{\iO}} = H(e^\iO)
\) dess frekvenssvar. Det är också bara \(
    H(z)
\) på enhetscirkeln och \(
    H(e^\iO)
\) är ju bara DTFT som är kontinuerlig i \(
    \Omega
\) och periodisk med period \(
    2\pi
\). 

Då är \(
    \abs{H(e^\iO)} 
        = b_0 \cdot \frac{\prod_{k=1}^M \abs{(e^\iO - c_k)} }{\prod_{k=1}^N \abs{(e^\iO - d_k)}} 
\). 

Låt \(
    \phi = \arg\{e^\iO - d_k\}
\). 

\begin{ex}
    Vi har ett FIR-filter som beskrivs av differensekvationen \(
        y[n] = x[n] + x[n-2]
    \).

    z-transformen ger \(
        Y(z) = X(z) z^{-2} + X(z) = X(z) (1+z^{-2})
    \) och därmed är \(
        H(z) = \frac{Y(z)}{X(z)} = 1 + z^{-2} = \frac{z^2+1}{z^2}  
    \). Nollställen här är de \(
        z
    \) som uppfyller \(
        z^2 + 1 = 0
    \), d.v.s.\ \(
        z = \pm i = e^{\pm i \frac{\pi}{2} }
    \). Polerna är \(
        z^2 = 0
    \).

    Frekvenssvaret fås av att sätta \(
        z
    \) i z-transformen till \(
        e^\iO
    \). Då får vi \(
        H(e^\iO) = 1 + e^{-2\iO}
            = e^{-\iO}(e^\iO + e^{-\iO})
            = 2 e^{-\iO} \cos \Omega
    \).

    \(
        \abs{H(e^\iO)}
    \) är då \(
        2 \abs{\cos \Omega}
    \).
\end{ex}

\end{document}