\documentclass[a4paper]{article}
\usepackage{../flpack}

\begin{document}

\providecommand\fname{}
\renewcommand\fname{19-09-17}

\subsection{Diskret Fouriertransform}
Idén för att göra den diskret är sampling. Det innebär att man väljer en 
upplösning \(
    N
\) så att man tar värdet på funktionen i punkterna \(
    0, \epsilon, 2\epsilon, \dots
\) där \(
    \epsilon = \bif{T}{N} 
\). Alltså approximerar vi \(
    f
\) m.h.a.\ \(
    f(\bif{kT}{N} )
\) där \(
    N
\) går från \(
    0
\) till \(
    N-1
\).

\begin{ansats}
    \(
        D[n] = \bif{1}{T} \bs_{k=0}^{N-1} \bi_\frac{kT}{N}^{\frac{k+1}{T} N} f\left(\frac{kT}{N}\right) e^{-i\oz n \frac{kT}{N}} \dd t
        = \bif{1}{N} \bs_{k=0}^{N-1} f\left(\frac{kT}{N}\right) e^{-i \frac{2\pi}{N}kn}
    \)  
\end{ansats}

\(
    D_x[n] = \inv{N} \bs_{k=0}^{N-1} x[k] e^{-i \frac{2\pi}{N}kn}
\) 

\subsubsection{Syntes}
\(
    x[k] = \bs_{k=0}^{N-1} D_x[n] e^{i \frac{2\pi}{N}kn}
\) 

\dquote{Ni fnissar på tok för lite}{Mattesnubben om den k:te punkten}
\dquote{Jag har approximerat allt med en konstant, så det är ganska grovt.}{Mattesnubben}

Anledning till att det här gäller är följande:
\paragraph{Steg 1}
\(
    1-z^{N} = (1-z)(1+z+z^2+\cdots + z^{N-1}) = 1+z+z^2+\cdots+z^{N-1}-z-z^2-\cdots-z^{N-1}-z^{N}
\) 

\paragraph{Steg 2}
\(
    z_k = e^{\frac{2\pi i}{N}k}
\) uppfyller \(
    z_k^N = 1
\) 

\paragraph{Steg 3}
\(
    1 + z_k + z_k^2 + \cdots \ z_k^{N-1}
\) är antingen \(
    N
\) när \(
    k = 0
\) och \(
    0
\) annars.

\(
    HL = \bs_{n=0}{N-1} D_x[n] z_k^n = \inv{N}\bs_{n,l=0}^{N-1} x[l] z_{-l}^n z_k^n
    = \inv{N}\bs_{l=0}^{N-1}\left(\bs_{n=0}^{N-1} x[l] z_{k-l}^n\right)
    = \inv{N}\bs_{l=0}^{N-1}x[k] = x[k]
\) 

\subsection{Fouriertransformen}
Här kollar vi på signaler \(
    f(t), -\infty < t < \infty
\).

För \(
    T >> t
\) kan vi skriva \(
    f(t) = \inv{2\pi}\bs_{k=-\infty}^\infty \bif{2\pi}{T} \left(\bi_{-\frac{T}{2} } ^{\frac{T}{2} } f(u) e^{-i \frac{2\pi k}{T}u} \dd u \right) e^{i \frac{2\pi k}{T}t}
\) 

\(
    \omega = \bif{2\pi k}{T} 
\) 

\begin{defn}[Fouriertransformen]
    \(
        F(i\omega) = \bi_{-\infty}^{\infty} f(u) e^{j\omega u} \dd u
    \) 
\end{defn}

\begin{sats}[Rekonstruktion av signal från Fourierserie]
    \(
        f(t) = \inv{2\pi}\bi_{-\infty}^\infty \bif{2\pi}{T} F(i\omega) e^{i j\omega t}
    \) 
\end{sats}

Låt \(
    y = h * x
\) vara ett LTI-system. Då är \(
    H(i\omega) 
\) överföringsfunktionen och om vi tar \(
    x(t) = e^{i\omega t}
\) så är \(
    y (t) = h * x(t) = \infint h(u) e^{i\omega (t-a)} \dd u 
    = e^{i\omega t} \infint h(u) e^{-i\omega u} \dd u
    = e^{i\omega t}H(i\omega)
\). 

När man tar \(
    h * \cos(\omega t + \theta) = \abs{H(i \omega)} \cos(\omega t + \theta + \angle H(i\omega))
\) 

\dquote{Keepin' it real}{Mattesnubben om cosinus-funktioner}

\begin{ex}
    Låt \(
        f_1(t) = e^{-at}u(t)
    \) där \(
        a > 0
    \). Räkna ut \(
        F_1(i\omega)
    \).

    \(
        F_1(i\omega) = \infint e^{-at} u(t) e^{-i\omega t} \dd t 
        = \infint e^{-(a+i\omega)t} \dd t
        = \left[\bif{e^{-(a+i\omega)t}}{-(a+i\omega)} \right]_{t=0}^\infty 
        = \inv{a+i\omega}
    \).
\end{ex}

\begin{ex}
    Låt \(
        f_2(t) = e^{-a\abs{t}}
    \). Räkna ut \(
        F_2
    \).

    Notera att \(
        f_2 = f_1(t) + f_1(-t)
    \) där \(
        f_1
    \) kommer från förra exemplet.

    Låt \(
        f_{op}(t) = f(-t)
    \). \(
        F_{op}(\io) = \infint f(-t) e^{-\io t} \dd t
        = (u = -t) = \infint f(u) e^{-i(-\omega u)} = F(-\io)
    \). Alltså är \(
        F_2(\io) = F_1(\io) + F_1(\io) = \inv{a+\io} + \inv{a-\io} = \bif{2a}{a^2+\omega^2} 
    \) 
\end{ex}

\begin{sats}[Linearitet av Fouriertransformer]
    Fouriertransformen är linjär, d.v.s.\ om \(
        f = f_1 + \lambda \cdot f_2 
    \) är \(
        F = F_1 + \lambda \cdot F_2
    \).

    \begin{proof}
        Visas lätt från definitionen av Fouriertransformen och faktumet att 
        integrering är linjärt.
    \end{proof}
\end{sats}

\subsection{\(
    \delta(t)
\) }

\(
    \infint \delta(t-a) x(t) \dd t = x(a)
\) 

Fouriertransformen av \(
    \delta(t)
\) är då \(
    F(\io) = \infint \delta(t) e^{-\io t} \dd t
    = e^{-\io a}
\).

\subsection{Transformpar}
Ett transformpar är paret \(
    (f(t), F(\io))
\), d.v.s.\ \(
    f(t)
\) svarar mot \(
    F(\io)
\). Exempelvis svarar \(
    e^{-at}u(t)
\) mot \(
    \inv{a+\io}
\).

\begin{ex}
    Låt \(
        f(t) = \bif{2a}{a^2+t^2} 
    \). \(
        F(t)
    \) är då 
    \begin{align*}
        F(t) &= \infint \bif{2a}{a^2+t^2} e^{-\io t} \dd t \\
        &= 2\pi \cdot \inv{2\pi} \infint \bif{2a}{a^2+t^2} e^{j(-\omega)t} \dd t\\
        &= \text{ (via syntesformeln) } \\
        &= 2\pi e^{-a\abs{-\omega}} \\
        &= 2\pi e^{-a\abs{\omega}}.
    \end{align*}
\end{ex}

\dquote{Lägg inte in någon magi i detta, bara lite fantasi.}{Mattesnubben}

\begin{påst}
    \(
        (f(t), F(\io))
    \) är ett transformpar \(
        \iff (F(-it), 2\pi f(\omega))
    \) är ett transformpar.
\end{påst}

\begin{ex}
    \(
        (\delta(t-a), e^{-\io a})
    \) är ett transformpar.

    Alltså är \(
        (e^{ita}, 2\pi \cdot \delta(\omega-a))
    \) ett transformpar.

    En ren frekvens \(
        a
    \) ger en Fouriertransform som är \( 
        \delta(t-a)
    \).
\end{ex}

Följande gäller: \begin{align*}
    \inv{2\pi} \infint F(\io) e^{\io t} \dd \omega
    &= \inv{2\pi} \infint \infint F(u) e^{\io (u-t)} \dd u \dd \omega\\
    &= \inv{2\pi} \infint f(u) \infint e^{\io u} e^{\io t} \dd \omega \dd u\\
    &= \inv{2\pi} \infint f(u) 2\pi \delta(t-u) \dd u\\
    &= f(t).
\end{align*}

\subsection{Periodisk återblick}
Låt \(
    x
\) vara en \(
    T
\)-periodisk signal. Då vet vi att vi kan skriva \(
    x(t) = \bs_{k=-\infty}^\infty c_k e^{\io_0 k t}
\). Lineariteten av Fouriertransformen ger då \(
    X(\io) = \bs_{k=-\infty}^\infty c_k \delta(\omega - k \omega_0)
\).

\subsection{Faltning}
Låt \(
    x_1, x_2
\) vara signaler. Då är \(
    x_1 * x_2 = \infint x_1(u) x_2(t-u) \dd u
\). 

\begin{påst}
    \(
        x_1 * x_2 
    \) är ett transformpar med \(
        X_1X_2
    \).

    \begin{ex}
        Låt \(
            x_1 = \delta(t-b), x_2 = \delta(t-a)
        \). 
        
        Då är \(
            x_1*x_2(t) = \infint \delta(u-b) \delta(t-u-a) \dd u
            = \delta(t-b-a)
        \).

        Fouriertransformen av \(
            x_1*x_2 = e^{-i(a+b)t} = e^{-iat} e^{ibt} = X_1(\io) X_2(\io)
        \).
    \end{ex}

    \begin{proof}
        \begin{align*}
            \infint x_1*x_2 e^{-\io t} \dd t 
            &= \infint \infint x_1(u) x_2(t-u) e^{-\io t} \dd u \dd t\\
            &= \text{ (Variabelbyte där } v = t-u \text{) } \\
            &= \infint \infint x_1(u) x_2(v) e^{-\io(v+u)} \dd u \dd v\\
            &= \infint \infint x_1(u) e^{-\io u} x_2(v) e^{-\io v} \dd u \dd v\\
            &= \infint x_1(u) e^{-\io u} \dd u \cdot \infint x_2(v) e^{-\io v} \dd v\\
            &= X_1(\io) \cdot X_2(\io)
        \end{align*}
    \end{proof}
\end{påst}

\begin{påst}
    \(
        x_1(t)x_2(t) 
    \) är ett transformpar med \(
        2\pi X_1*X_2(-\io)
    \). 
\end{påst}

\(
    x(t)\delta(t-a) = x(a)\delta(t-a)
\) ger att \(
    H(\io) X(\io) = \bs_{k=-\infty}^\infty c_k H(\io_0 k) \delta(\omega - k\omega_0)
\).

\end{document}