\documentclass[a4paper]{article}
\usepackage{../flpack}

\begin{document}

\providecommand\fname{}
\renewcommand\fname{19-10-03}

\subsection{Kort repetition}
Kom ihåg att för ett LTI-system gäller att för ett system med insignal \(
    x(t) = \sin(\omega t)
\) och frekvenssvar (FT av impulssvaret) \(
    G(\io)
\) får vi utsignalen \(
    y(t) = \abs{G(\io)} \sin(\omega t + \arg{G(\io)})
\) i stationärtillstånd. Om man vill ha hela förloppet kan man 
Laplacetransformera och räkna som vanligt.

\subsection{Bodediagram/Bode plots}
En grafisk presentation av frekvenssvaret:

Innehåller två delar:
\begin{itemize}
    \item Amplituddiagram, ofta med logaritmisk frekvens- och amplitudskala 
    (i \si{\deci\bel}). \(
        \abs{G(\io)} = 20 \cdot \log_{10}(\abs{G(\io)})
    \) 
    \item Fasdiagram, ofta med logaritmisk frekvensskala men linjär fasskala 
    (d.v.s.\ plottar mot \(
        \arg{G(\io)}
    \))
\end{itemize}

För att konstruera dessa diagram utgår vi ifrån en överföringsfunktion \(
    G(s)
\) och sätter \(
    s=\io
\) så att vi får \(
    G(\io)
\). Sedan faktoriserar vi överföringsfunktionen \(
    G(s)
\) så att \(
    G(s) = \frac{C_1(s)C_2(s)\cdots C_M(s)}{D_1(s)D_2(s)\cdots C_N(s)} 
\). Faktorerna \(
    C_k(s)
\) och \(
    D_k(s)
\) är antingen \(
    k
\) (en konstant), \(
    s
\) (derivering i täljaren, integrering i nämnaren), \(
    1 + \frac{s}{\omega_1} 
\) (förstagradsfaktor p.g.a.\ en reell rot) eller \(
    1 + s \frac{2a}{\omega_2} + \frac{s^2}{\omega_2^2} 
\) (andragradsfaktor p.g.a.\ komplexa rötter).

\subsubsection{Frekvenssvarets belopp}
\(
    \abs{G(\io)} = \frac{\abs{C_1(s)}\abs{C_2(s)}\cdots \abs{C_M(s)}}{\abs{D_1(s)}\abs{D_2(s)}\cdots \abs{C_N(s)}}
\). I \si{\deci\bel} är det \(
    \abs{G(\io)}_{\si{\deci\bel}}
        = \abs{C_1(\io)}_{\si{\deci\bel}} + \abs{C_2(\io)}_{\si{\deci\bel}} + \cdots + \abs{C_M(\io)}_{\si{\deci\bel}}
            - \abs{D_1(\io)}_{\si{\deci\bel}} - \abs{D_2(\io)}_{\si{\deci\bel}} - \cdots - \abs{D_N(\io)}_{\si{\deci\bel}}
\) och deras fasbidrag fås som \(
    \arg{G(\io)} =
    = \arg{C_1(\io)} + \arg{C_2(\io)} + \cdots + \arg{C_M(\io)}
        - \arg{D_1(\io)} - \arg{D_2(\io)} - \cdots - \arg{D_N(\io)}
\). Notera att superpositionen av bidragen från varje delfaktor ger både \(
    \abs{G(\io)}_{\si{\deci\bel}}
\) och \(
    \arg{G(\io)}
\).

Studera \(
    \eval{\frac{1}{1 + \frac{s}{\omega_1}}}_{s=\io} = \frac{1}{1+ i \frac{\omega}{\omega_1} } 
\). Dess belopp är \(
    \frac{1}{\sqrt{1 + \frac{\omega}{\omega_1}^2}} = \abs{G_1(\io)}
\). Bodediagrammet blir då \f

Argumentet för den är \(
    \arg{G_1(\io)} = - \arctan(\frac{\omega}{\omega_1} )
\). Bodediagrammet blir \f

Lutningen på amplituddiagrammet är \(
    -20~\frac{\si{\deci\bel}}{\text{dekad}} 
\).

\subsection{Fourierrepresentationer}
Ordlista: CT -- Continuous time, DT -- Discrete time, FS -- Fourier series, 
FT -- Fourier transform

En tabell som beskriver vad vi använder i olika fall: 

\begin{tabular}{lll}
    Signal & Periodisk                                & Ickeperiodisk     \\ \hline
    CT     & CTFS (\(c_k\) eller \(a_k\) och \(b_k\)) & CTFT (\(X(\io)\)) \\
    DT     & Vi väntar med denna.                     & DTFT             
\end{tabular}

\begin{påm}[Kort repetition av sampling]
    För \(
        x_p(t) = p(t) \cdot x(t) 
    \) är \(
        X_p(\io) = \inv{T} \sum_{k=-\infty}^\infty X(i(\omega-k\omega_s))
    \) där \(
        \omega_2 = \frac{2\pi}{T} 
    \) och \(
        p(t) = \sum_{n=-\infty}^\infty \delta(t-nT)
    \).
\end{påm}

Om vi Fouriertransformerar \(
    x_p(t)
\) direkt får vi 
\begin{align*}
    X_p(t) &= \infint x_p(t) e^{-\io t} \dd t \\
           &= \infint \qty( \sum_{n=-\infty}^\infty x(t) \cdot \delta(t-nT) ) e^{-\io t} \dd t \\
           &= \sum_{n=-\infty}^\infty \infint x(t) \cdot \delta(t-nT) e^{-\io t} \dd t \\
           &= \{\qq*{Får endast bidrag} \neq 0 \qq{vid impuls, d.v.s.\ vid} t = nT\} \\
           &= \sum_{n=-\infty}^\infty x(nT) e^{-\io nT} \infint \delta(t-nT) \dd T \\
           &= \sum_{n=-\infty}^\infty x(nT) e^{-\io nT} \\
           &= \{ \qq*{Låt} x(nT) = x[n] \qq{och} \omega T = \Omega \} \\
           &= \sum_{n=-\infty}^\infty x[n]e^{-i \Omega n}
\end{align*}
vilket är Fouriertransformen för en ickeperiodisk diskret signal och 
vi skriver den som \(
    X(e^{i\Omega }) = \sum_{n=-\infty}^\infty x[n]e^{-i \Omega n}
\). Kallas Diskret Tid FourierTransform och förkortas DTFT. Kursboken använder 
beteckningen \(
    X(\Omega)
\) men det är \enquote{bättre} med konventionen ovan eftersom det matchar 
bättre mot z-transformen senare i kursen.

\subsubsection{Egenskaper}
\(
    X(e^{\iO})
\) är\dots
\begin{itemize}
    \item kontinuerlig i \(
        \Omega
    \).

    \item periodisk i \(
        \Omega
    \) eftersom \(
        X(e^{i(\Omega + 2 \pi k)}) 
            = \sum_{n=-\infty}^\infty x[n] e^{-i(\Omega + 2 \pi k) n}
            = \sum_{n=-\infty}^\infty x[n] e^{-i\Omega n} \cdot e^{-i2\pi k n}
            = \sum_{n=-\infty}^\infty x[n] e^{-i\Omega n} 
            = X(e^{\iO})
    \) 
\end{itemize}

Alltså är \(
    x[n]
\) innan DTFT både diskret och icke-periodisk, men \(
    X(e^{\iO})
\) är både kontinuerlig och periodisk.

Kom ihåg hur Fouriertransformen för vårt viktade impulståg \(
    x_p(t)
\) såg ut: \f

Den är också periodisk, men i \(
    \omega
\) och med \enquote{perioden} \(
    \omega = \omega_s
\). Men \(
    \Omega = \omega T
\) vilket vid \(
    \Omega = 2 \pi 
\) ger \(
    \omega = \frac{2\pi}{T} = \omega_s
\) eftersom \(
    T
\) är samplingsintervallet. \(
    \Omega = 2\pi
\) motsvarar samplingsvinkelfrekvensen. \(
    e^{\iO n} = \cos(\Omega n) + i \sin(\Omega n)
\). Enheten för \(
    \Omega
\) är då \si{\radian} alt \(
    \frac{\si{\radian}}{\text{sampel}}
\). 

\subsection{Syntesekvation/Invers DTFT}
\(
    x[n] = \int_{2\pi} X(e^{\iO}) e^{i\Omega n} \dd \Omega
\).  

\end{document}