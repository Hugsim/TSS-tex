\documentclass[a4paper]{article}
\usepackage{../flpack}

\begin{document}

\providecommand\fname{}
\renewcommand\fname{19-09-18}

\subsection{Egenskaper hos Fouriertransformen}
\dquote{Det är naturligtvis lite mer delikat än så.}{Pokèmon \#109}

Om \(
    \infint \abs{f(t)} \dd t < \infty
\) är alltid \(
    F(\io) 
\) kontinuerlig och \(
    \lim_{\abs{\omega} \to \infty} F(\io) = 0
\). 

\begin{ex}
    Om \(
        f = \delta
    \) saknar \(
        \int \abs{f} \dd t 
    \) mening. Trots det är \(
        F(\io) = 1
    \) kontinuerlig, men \(
        \lim_{\abs{\omega} \to \infty} = 1 \neq 0
    \). 
\end{ex}

\begin{ex}
    \(
        f(t) = e^{-\abs{a}t}
    \) bildar transformpar med \(
        F(\io) = \bif{2a}{a^2+\omega^2} 
    \). 

    \(
        f(t) = e^{-at}u(t)
    \) bildar transformpar med \(
        F(\io) = \inv{a+\io}
    \). 
\end{ex}

\subsection{Räkneregler för Fouriertransformen}
\begin{sats}[Linearitet av Fouriertransformen]
    Om \(
        A f_1(t) + Bf_2(t)
    \) är dess Fouriertransform \(
        AF_1(\io) + B F_2(\io)
    \).
\end{sats}

\begin{sats}[Tidsskalning Fouriertransform]
    Om \(
        f(at) 
    \) när \(
        a \neq 0
    \) är dess Fouriertransform \(
        \inv{\abs{a}} F(\frac{\io}{a})
    \).

    \begin{proof}
        \(
            \infint f(at) e^{-\io t} \dd t = \{u = at \implies \dd u = a \dd t\} 
            = \inv{\abs{a}} \infint f(u) e^{-j\frac{\omega}{a}u} \dd u
        \).
    \end{proof}
\end{sats}

\begin{sats}[Tidsskifte Fouriertransform]
    Om \(
        f(t+a)
    \) är dess Fouriertransform \(
        e^{\io a} F(\io)
    \).

    \begin{proof}
        \(
            \infint f(t+a) e^{-\io t} \dd t = \{u = t+a\} 
            = \infint f(u) e^{\io a} e^{-\io u} \dd u
            = e^{\io a} F(\io)
        \).
    \end{proof}
\end{sats}

\begin{ex}
    Låt \(
        f_1(t) = 
        \left\{\begin{matrix}
            \inv{2a}, \abs{t} \leq a\\ 
            0, \text{ annars} 
        \end{matrix}\right.
    \).

    Då är \(
        F_1(\io) = \infint f_1(t) e^{-\io t} \dd t 
        = \inv{2a} \int_{-a}^a e^{-\io t} \dd t
        = \inv{2a} = \left[\frac{e^{-\io t}}{-\io}\right]_{t=-a}^a
        = \frac{ e^{\io a} - e^{-\io a} }{2i} \inv{\omega a} 
        = \frac{\sin(\omega a)}{\omega a}
    \).

    Låt \(
        x
    \) vara en signal. Då är \(
        f_1*x = \infint f_1(s) x(t-s) \dd s = \inv{2a} \int_{-a}^a x(t-s) \dd s
    \) vilket är medelvärdet av \(
        x(t)
    \) i punkten \(
        t
    \) när man går \(
        a 
    \) steg till både höger och vänster. Inses rätt lätt om man ritar upp det.

    \(
        f_1*x 
    \) bildat transformpar med \(
        \frac{\sin(\omega a)}{\omega a} X(\io)
    \).
\end{ex}

\begin{sats}[Derivata och Fouriertransform]
    Om \(
        f(t)
    \) är en signal bildar \(
        \dv{t} f(t)
    \) ett transformpar med \(
        \io F(\io)
    \). 

    \begin{proof}
        \(
            \infint \dv{t} f(t) e^{-\io t} \dd t
            = 
        \) (med partiell integration och antagandet att 
        \(
            f
        \) går mot \(
            0
        \)  i oändligheten) \(
            = - \infint f(t) \left(\dv{t} e^{-\io t}\right)\dd t
            = \io \infint f(t) e^{-\io t} \dd t
        \) 
    \end{proof}
\end{sats}

\begin{ex}
    Tillbaka till förra exemplet, vad är \(
        f_1'
    \)?

    Jo, \(
        f_1' 
    \) kan definieras av att \(
        \infint f'(t)x(t) \dd t = -\infint f_1(t) x'(t) \dd t
        = -\inv{2a} \int_{-a}^a x'(t) \dd t 
        = \inv{2a} (x(-a) - x(a)) 
        = \infint \inv{2a} (\delta(t+a) - \delta(t-a)) x(t) \dd t
    \). Alltså är \(
        f_1' = \inv{2a} (\delta(t+a) - \delta(t-a))
    \). 

    Om man tittar på grafen för \(
        f_1
    \) kan man, om man tänker på vad derivatan måste vara i varje punkt,
    se att detta är rimligt.

    Vi vill kolla att detta stämmer. \(
        f_1' = \inv{2a} (\delta(t+a) - \delta(t-a))
    \) bildar transformpar med \(
        \inv{2a} (e^{\io a} - e^{-\io a})
        = \io \inv{2ia\omega} (e^{\io a} - e^{-\io a})
        = \io \inv{a\omega} \sin(a\omega)
        = \io F_1(\io)
    \), vilket är rimligt med tanke på tidigare räkneregler.
\end{ex}

\begin{sats}[Fouriertransform för \(t \cdot f(t)\)]
    Om vi har en signal \(
        t\cdot f(t)
    \) bildar det ett transformpar med \(
        i \dv{\omega} F(\io)
    \).

    \begin{proof}
        \(
            \infint t\cdot f(t) e^{-\io t} \dd t 
            = \infint f(t) t e^{-\io t} \dd t 
            = \infint f(t) i \dv{\omega} e^{-\io t} \dd t
            = i \infint f(t) \dv{\omega} e^{-\io t} \dd t
            = i \dv{\omega} \infint f(t) e^{-\io t} \dd t
        \).
    \end{proof}
\end{sats}

\dquote{Låt oss vara lite casual}{Pokémon \#109 om räkneregler}

VARNING: I boken skrivs \(
    F(\omega)
\) för Fouriertransformen i kapitel \(
    < 6
\) men \(
    F(\io)
\) i kapitel \(
    \geq 6
\).

\begin{ex}
    Låt \(
        f_2(t) = e^{-at^2}, a > 0
    \). Vad är Fouriertransformen \(
        F_2
    \)?

    Vi ska ta en systemapproach istället för att göra jobbiga integraler.

    \(
        \dv{t} f_2(t) = -2t e^{-at^2} = -2atf_2(t)
    \), d.v.s.\ \(
             (\dv{t} + 2at)f_2(t) = 0 
        \iff (\io + 2ai \dv{\omega})F_2 = 0
        \iff (\dv{\omega} + \inv{2a} \omega) F_2 = 0
        \iff F_2(\omega) = Ce^{-\frac{\omega^2}{4a}}
        \implies C = F_2(\omega=0)\cdot e^{\frac{0^2}{4a}} 
        = F_2(0)
        = \infint e^{-at^2}\dd t
        = \sqrt{\frac{\pi}{a}}
    \). Alltså bildar \(
        e^{-at^2}
    \) transformpar med \(
        \sqrt{\frac{\pi}{a}} e^{-\frac{\omega^2}{4a}}
    \). 
    \begin{anm}
        \(
            f_2
        \) avtar superexponentiellt när \(
            t \to \pm \infty
        \), d.v.s. \(
            f_2 
        \) inte en lösning till en stabil LTI.
    \end{anm}
\end{ex}

\dquote{Den går snabbare än örnen mot noll.}{Pokèmon \#109}

Vad betyder egentligen \(
    \dv{t}
\)?

Jo, \(
    \dv{t} (h*x) = \dv{h}{t} * x = h * \dv{x}{t}
\). \(
    \dv{t}
\) är också tidsinvariant, vilket betyder att det måste ges av någon faltning.
Vad för \(
    h
\) uppfyller att \(
    h * x ? \dv{t} x
\)?

\(
    x(t) = \int x(u) \delta(t - u) \dd u = x * \delta(t) = \delta * x(t)
\). Då är \(
    \dv{t} x = \dv{t}(\delta * x) = (\dv{t} \delta) * x
\). \(
    \delta
\) definieras av att \(
        \infint \delta(t) x(t) \dd t = x(0)
\), alltså måste \(
    \infint \delta'(t)x(t) \dd t = - \infint \delta(t) x'(t) \dd t
\) så att \(
    \delta'
\) definieras av \(
    \infint \delta'(t) x(t) \dd t = - x'(0)
\). Då är \(
    \dv{t} x = \delta' * x
\).

\begin{sats}[Sambandet mellan \(\dv{t} \text{ och } \delta'\)]
    \(
        \dv{t} x = \delta' * x
    \) 

    \begin{proof}
        Se ovan.
    \end{proof}
\end{sats}

\begin{ex}
    Vad är Fouriertransformen av \(
        \delta'
    \)?

    \(
        \infint \delta'(t) e^{-\io t} \dd t 
        = \{ e^{-\io t} = x(t) \} 
        = -x'(0) 
        = -\io
    \) 
\end{ex}

\begin{ex}
    Skriv \(
        (-\dv[2]{t} + a^2)y = x
    \) som en LTI på formen \(
        y = h*x
    \).

    \(
        (-\dv[2]{t} + a^2)y 
    \) bildar transformpar med \(
        (-{(\io)}^2+a^2)Y = (\omega^2 + a^2)Y
    \).

    \(
        (-\dv[2]{t} + a^2)y = x \iff (\omega^2 + a^2)Y = X
        \iff Y = \inv{\omega^2+a^2}X
        \iff y = h * x
    \) för ett \(
        h
    \) s.a.\ \(
        H(\io) = \inv{\omega^2+a^2}
    \).

    \(
       e^{-at} 
    \) har en Fouriertransform \(
        \frac{2a}{a^2+\omega^2}
    \), då har \(
        h(t) = \inv{2a} e^{-at}
    \) överföringsfunktion \(
        H(\io) = \inv{a^2+\omega^2}
    \). 

    Då får vi från ursprungliga problemet \(
        h * x(t) = y(t) = \inv{2a} \infint e^{-a\abs{t-u}}x(u)\dd u
    \).
\end{ex}

\dquote{En pil säger mer än två ord i det här fallet}{Pokèmon \#109}

\begin{sats}
    Låt \(
        f
    \) vara en signal. Då har \(
        \int_{-\infty}^t f(\tau) \dd \tau
    \) Fouriertransform \(
        \frac{F(\io)}{\io} + \pi F(0) \delta(\omega)
    \). 

    \begin{proof}
        Idén bakom beviset är att \(
            \int_{-\infty}^t f(\tau) \dd \tau 
            = \infint f(\tau) u(t - \tau) \dd \tau
            = f * u
        \).

        Eftersom \(
            f * u 
        \) bildar transformpar med \(
            FU
        \) vill vi visa att det är lika med \(
            F(\io) \left( \inv{\io} + \pi \delta(\omega) \right)
        \). Vi vill alltså att \(
            U(\io) = \inv{\io} + \pi \delta(\omega)
        \).

        Eftersom \(
            u' = \delta 
        \) får vi att det bildar transformpar med \(
            \io U(\io) = 1 \implies U(\io) = \inv{\io}
        \).

        Man får i någon mån halva bidraget från den konstanta funktionen \(
            1
        \), vilket är \(
            2\pi\delta(\omega)
        \), alltså får vi en faktor \(
            \pi\delta(\omega)
        \) i lösningen.
    \end{proof}
\end{sats}

Om \(
    \infint f(\tau) \dd \tau = 0
\) måste \(
    F(0) = 0
\).

\subsection{Plancherels sats}
\begin{sats}
    Låt \(
        L^2(-\infty, \infty) = \{\text{ signaler } f \text{ med ändlig energi } \}
    \) Kom ihåg att energin \(
        E = \infint \abs{f(t)}^2 \dd t
    \). Kom också ihåg \(
        \lang f_1, f_2 \rang = \infint \overline{f_1(t)} f_2(t) \dd t
    \). Då är, för \(
        f_1, f_2 \in L^2(-\infty, \infty) 
    \), både \(
        F_1
    \) och \(
        F_2 \in L^2(-\infty, \infty) 
    \). Vidare är \(
        \infint \overline{f_1(t)} f_2(t) \dd t 
        = \inv{2\pi} \infint \overline{f_1(t)} f_2(t) \dd t
    \).
\end{sats}

\end{document}