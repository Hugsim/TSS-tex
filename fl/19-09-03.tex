\documentclass[a4paper]{article}
\usepackage{../flpack}

\begin{document}

\providecommand\fname{}
\renewcommand\fname{19-09-03}

I kursen studerar vi tekniska/fysikaliska system och deras egenskaper. Vi kan 
till exempel vara intresserade av hur de reagerar på olika exciteringar 
(insignaler). Vi begränsar oss till system med en in- och en utsignal. 
In- och utsignaler kan vara ex.\ kraft, tryck, spänning, ström o.s.v. 
Vi använder oss av matematiska modeller för att beskriva signaler 
(funktioner) och system (ekvationer). 

\subsection{Klassificering av signaler}
\subsubsection{Diskret/Kontinuerlig signal}
Kontinuerlig (tid) signal \(
    x(t) \text{ där } t \in \mathbb{R}
\)

Diskret (tid/oberoende variabel) signal \(
    x[n] \text{ där } n \in \mathbb{Z}
\) 

Kontinuerlig amplitud \(
    x(t), x[n] \in \mathbb{R}
\) 

Diskret amplitud \(
    x(t), x[n] 
\) kvantiserad

Digital signal betecknar vanligen en signal som är diskret både i tid och 
amplitud.

\subsubsection{Jämn/Udda signal}
Jämn signal \(
    \forall t, x(t) = x(-t)
\) kallas vanligtvis \(
    x_e(t)
\) 

Udda signal \(
    \forall t, x(t) = -x(-t)
\) kallas vanligtvis \(
    x_o(t)
\) 

\begin{sats}
En godtycklig signal \(
    x(t)
\) kan alltid delas upp i en jämn signal \(
    x_e(t)
\) och en udda signal \(
    x_o(t)
\). Den jämna delen \(
    x_e(t) = \half (x(t) + x(-t))
\) och den udda delen \(
    x_o(t) = \half (x(t) - x(-t))
\) 
\begin{proof}
    \(
        x_e(-t) = \half (x(-t) + x(t)) = x_e(t)
    \) och \(
        x_o(-t) = \half (x(-t) - x(t)) = - \half (x(t) - x(-t)) = - x_o(t)
    \). Summan \(
        x_e(t) + x_o(t) = \half (x(t) + x(-t) + x(t) - x(-t)) = x(t)
    \).
\end{proof}

\end{sats}

\subsubsection{Periodisk signal}
Periodisk signal \(
    \forall t, x(t) = x(t+T)
\) där \(
    T
\) är en konstant period för signalen.

Exempelvis sinusformad signal, fyrkantsvåg, triangelvåg o.s.v.

Periodisk diskret signal \(
    \forall n, x[n] = x[n+N] 
\) där \(
    N \in \mathbb{Z}^+ 
\) och konstant.

\subsubsection{Deterministisk (Känd, förutsägbar)/Slumpmässig (Stokastisk) signal}
En signal kallas slumpmässig om den inte kan förutsägas helt.

\begin{defn}[Energisignal]
    Låt den totala energin vara \(
        E = \blim_{T \to \infty} \bi_{-\frac{T}{2}}^{\frac{T}{2}} \abs{x(t)}^2 \dd t
    \) alt.\ \(
        E = \bs_{n=-\infty}^{\infty} \abs{x[n]}^2
    \). Om \(
        0 < E < \infty
    \) är x en energisignal.
\end{defn}

\begin{defn}[Effektsignal]
    Låt medeleffekten \(
        P = \blim_{T \to \infty} \inv{T} \bi_{-\frac{T}{2}}^{\frac{T}{2}} \abs{x(t)}^2 \dd t
    \) alt.\ \(
        P = \blim_{T \to \infty} \inv{2N+1} \bs_{n=-\infty}^N \abs{x[n]}^2
    \). Om \(
        0 < P < \infty
    \) är \(
        x
    \) en effektsignal.
\end{defn}

Notera att för en energisignal gäller \(
    P \to 0
\) eftersom \(
    E < \infty
\) och på samma sätt gäller att för en effektsignal går \(
    E \to \infty 
\). 

\subsection{Signalmanipulering}
\subsubsection{Amplitudskalning}
\(
    y[n] = ax[n]+b
\) där vi exempelvis kan kalla \(
    a
\) för förstärkning och \(
    b
\) inom elektronik för DC-skift. \(
    a, b
\) är konstanter. På samma sätt för kontinuerliga signaler \(
    y(t) = ax(t) + b
\). 

\subsubsection{Ändra tidsskala}
\(
    y(t) = x(at), a \in \realn
\) och \(
    y[n] = x[kn], k \in \intn
\) 

\subsubsection{Spegling}
\(
    \forall t, y(t) = x(-t)
\) och \(
    \forall n, y[n] = x[-n]
\) 

\subsubsection{(Tids)skift}
\(
    y(t) = x(t-t_0)
\) och \(
    y[n] = x[n-n_0]
\) 

\subsubsection{Samtida operationer}
Låt \(
    x(t)
\) vara en signal.

Ändra tidsskala så att \(
    t \to at \implies x(at)
\) sen en tidsskift \(
    t \to t-t'_0 \implies x(a(t-t'_0)) = x(at-at'_0)
\). Nu ändrar vi först tidsskift \(
    t \to t-t_0 \implies x(t-t_0)
\) och sen tidsskalan \(
    t \to at \implies x(at-t_0)
\). Notera att det är blir olika beroende på i vilken ordning det tas i.
För att det ska vara lika måste \(
    at'_0 = t_0
\).

\subsection{Signalmodeller}
\subsubsection{Komplex exponential (kontinuerlig)}
\(
    x(t) = Ce^{at}
\) där \(
    C, a, x \in \compn
\). Komplexa tal förekommer oftast inte i fysikaliska system men är mycket
användbara som matematiska modeller. Den fysikaliska signalen kan fås
ur \(
    \Re{x(t)} \text{ eller } \Im{x(t)}
\). Jämför \(
    j\omega
\)-metoden (phasors) för beräkning av stationära växelströmskretsar.

\subsection{Lite repetition av komplexa tal}

\(
    \cos(\theta) = \bif{e^{\ji\theta} - e^{-\ji\theta}}{2} 
\) och \(
    \sin(\theta) = \bif{e^{\ji\theta} - e^{-\ji\theta}}{2j} 
\).

Tidsvariation \(
    \theta = \omega t
\) 

\subsection{Några olika fall}
\subsubsection{Fall 1}
Anta \(
    C, a \in realn
\). Då är \(
    x(t) = Ce^{at}
\). För \(
    a < 0 \text{ och } C > 0 
\) beskriver \(
    x
\) ett exponentiellt avtagande förlopp.
För \(
    a > 0 
\) beskriver \(
    x
\) ett exponentiellt stigande förlopp.
Om \(
    a = 0
\) är \(
    x(t) = C
\).

\subsubsection{Fall 2}
\(
    a, C \in \compn
\) och \(
    \Re{a} = 0
\). Låt \(
    a = \ji\omega_0
\) och \(
    C = A e^{\ji\Phi} (= A \angle \Phi)
\). \(
    x(t) = A e^{\ji\Phi} e^{j\omega_0t} = Ae^{j(\omega_0t+\Phi)} 
    = A \cos(\omega_0t+\Phi) + \ji A \sin(\omega_0t+\Phi)
\). \(
    \Re{x(t)} 
\) är sinusformad med amplitud \(
    A
\) och fasförskjutning \(
    \Phi
\). Det är en odämpad sinusformad signal.

\subsubsection{Fall 3}
\(
    C, a \in \compn
\). Låt \(
    C = Ae^{\ji\Phi} \text{ och } a = \sigma_0 + \ji\omega
\). Då är \(
    x(t) = Ae^{\ji\Phi} e^{(\sigma_0+j\omega_0)t} 
    = Ae^{\sigma_0t}e^{j(\omega_0t+\Phi)}
    = A e^{\sigma_0t}\cos(\omega_0t+\Phi) + \ji Ae^{\sigma_0t}\sin(\omega_0t+\Phi)
\). Om \(
    \sigma_0 < 0 
\) blir signalen en dämpad sinusformad signal. Om \(
    \sigma_0 > 0
\) blir signalen en anti-dämpad sinusformad signal. Om \(
    \sigma_0 = 0
\) är det bara fall 2 igen.

\subsection{Diskret exponential}
Låt \(
    x[n] = C a^n
\). Allmänt är \(
    C, a, x \in \compn
\) och \(
    n \in \intn
\).

\subsubsection{Fall 1}
\(
    C, a \in \realn
\). Då gäller \(
    x[n] = C a^n
\). För olika intervall ser graferna ut precis som man kan tänka sig.
\(
    a < 0
\) gör att tecknet på \(
    x[n]
\) växlar och är negativt för udda \(
    n
\). 

\subsubsection{Fall 2}
\(
    C, a \in \compn
\) men \(
    \abs{a} = 1
\). Låt \(
    a = e^{\ji\Omega_0} \text{ och } C = Ae^{\ji\Phi}
\). Då är \(
    A e^{\ji\Phi}e^{j\Omega_0n} = A e^{\ji(\Omega_0n+\Phi)} 
    = A \cos(\Omega_0n+\Phi) + \ji A\sin(\Omega_0n+\Phi)
\). \(
    x[n]
\) är då en diskret odämpad sinusformad signal. 

\subsubsection{Fall 3}
\(
    C, a \in \compn
\). Låt \(
    C = Ae^{\ji \Phi}
\) och \(
    a = e^{\Sigma_0} e^{\ji\Omega_0} = e^{\Sigma_0+\ji\Omega_0}
\). Då är \(
    x[n] = Ae^{\ji\Phi}e^{(\Sigma_0+\ji\Omega_0)n} 
    = A e^{\Sigma_0n} e^{\ji(\Sigma_0n+\Phi)}
\). \(
    \Sigma_0 
\) bestämmer om signalen blir dämpad eller anti-dämpad.

\end{document}