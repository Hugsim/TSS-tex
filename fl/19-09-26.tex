\documentclass[a4paper]{article}
\usepackage{../flpack}

\begin{document}

\providecommand\fname{}
\renewcommand\fname{19-09-26}

\subsection{Laplacetransformen}
Man brukar prata om en- och tvåsidiga Laplacetransformer. 
\subsubsection{Tvåsidig}
\begin{defn}[Tvåsidig Laplacetransform]
    Låt \(
        f(t), -\infty < t < \infty
    \) vara en signal. Då är Laplacetransformen \(
        F(s) = \infint f(t)e^{-st}\dd t
    \) där \(
        s \in \compn
    \), vilket typ är \(
        FT(f)
    \).
\end{defn}
    
Följande är det kausala fallet av den tvåsidiga transformen.
\begin{defn}[Ensidig Laplacetransform]
    Låt \(
        f(t)
    \) vara en kausal signal, d.v.s.\ \(
        f(t) = 0
    \) för \(
        t < 0
    \).

    Då är \(
        F(s) = \int_0^\infty f(t) e^{-st}\dd t
    \) där \(
        s \in \compn
    \).
\end{defn}

\begin{defn}[Exponentiellt begränsad]
    \(
        f
    \) är exponentiellt begränsad om efter något \(
        t_0
    \) beter den sig som en exponentiell funktion.

    Matematiskt säger vi att \(
        \exists \; c \in \realn, t_0 > 0 : \abs{f(t)} \leq e^{ct} \forall t > t_0
    \).
\end{defn}

\begin{ex}
    \(
        f(t) = \delta(t-a)
    \). Om \(
        a \geq 0
    \) är den kausal.

    För alla \(
        t_0 > a
    \) är \(
        f(t) = 0 \forall t > t_0
    \) och därmed \(
        \abs{f(t)} \leq e^{ct} \forall c \in \realn
    \). Alltså är den exponentiellt begränsad.
\end{ex}

\begin{påst}[När Laplace är definierat]
    Om \(
        f 
    \) är exponentiellt begränsad för något \(
        c \in \realn
    \) så är Laplacetransformen \(
        F(s)
    \) väldefinierad för \(
        \Re{s} > c
    \).

    \begin{proof}
        Tanken är att man visar att andra termen i \(
            F(s) = \int_0^{t_0} f(t) e^{-st} \dd t + \int_0^{t_0} f(t) e^{-st}\dd t
        \) är begränsad.
    \end{proof}
\end{påst}

\begin{ex}
    Låt \(
        f(t) = \delta(t-a)
    \) för något \(
        a > 0
    \).

    Då är \(
        F(s) = \int_0^\infty \delta(t-a) e^{-st} \dd t 
             = e^{-sa}
    \).
\end{ex}

\begin{ex}
    Låt \(
        f(t) = u(t)
    \). 

    Då är \(
        F(s) = \int_0^\infty e^{-st} \dd t = \inv{s}
    \).
\end{ex}

\begin{ex}
    Låt \(
        f(t) = tu(t)
    \).

    Då är \(
        F(s) = \int_0^\infty te^{-st} \dd t = \{\text{Partiell integrering}\} = \frac{n!}{s^{n+1}} 
    \).
\end{ex}

\begin{ex}
    Låt \(
        f(t) = \delta'(t-a)
    \) för något \(
        a > 0
    \). 

    Då är \(
        F(s) = \int_0^\infty \delta'(t-a) e^{-st} \dd t = se^{-sa}
    \).
\end{ex}

\begin{ex}
    Låt \(
        f(t) = e^{at}u(t)
    \). 

    Då är \(
        F(s) = \int_0^\infty e^{at} e^{-st} \dd t = \frac{1}{s-a} 
    \) om \(
        \Re{s} > \Re{a}
    \).
\end{ex}

\begin{påst}[Laplace av faltning]
    Om \(
        x_1, x_2
    \) är ensidiga gäller att Laplacetransformen av \(
        x_1 * x_2 = X_1X_2
    \).

    \begin{proof}
        Laplace av \(
            x_1*x_2 = \int_0^\infty (x_1 * x_2) e^{-st} \dd t
                    = \int_0^\infty \int_0^t x_1(\tau) x_2(t-\tau) e^{-st} \dd \tau \dd t
                    = \{ v = t-\tau \}
                    = \int_0^\infty \int_0^\infty x_1(\tau) x_2(v) e^{-s(\tau + v)} \dd \tau \dd v
                    = X_1(s)X_2(s)
        \).
    \end{proof}
\end{påst}

För en kausal LTI gäller att \(
    y = h*x \iff Y = HX
\). För att visa det här måste vi veta att Laplacetransformen av en signal är 
unik. 

\subsubsection{Egenskaper för Laplace}
Vi skriver \(
    \lap f(s) \coloneqq F(s)
\).

\begin{enumerate}
    \item \(
        \lap 
    \) är linjär, d.v.s.\ \(
        \lap(f_1+\lambda f_2) = \lap f_1 + \lambda \lap(f_2)
    \).

    \item \(
        f(t-a) \xmapsto{\lap} e^{-as} F(s)
    \) 

    \item \(
        e^{-zt} f(t) \xmapsto{\lap} F(s+z)
    \) 

    \item \(
        f'(t) \xmapsto{\lap} s \lap(f)(s) - f(0)
    \) eftersom \(
        \lap(f'(s)) = \int_0^\infty f'(t) e^{-st} \dd t
        = \left[ f(t) e^{-st} \right]_{t=0}^\infty - \int_0^\infty f(t) \cdot (-s) e^{-st} \dd t
        = f(\infty) e^{-s\infty} - f(0) + s \int_0^\infty f(t) e^{-st} \dd t 
        = sF(s) - f(0)
    \) 
\end{enumerate}

Om \(
    f
\) är en signal med  \(
    c < 0
\) är \(
    F(s) 
\) definierad för \(
    \Re(s) = 0
\). Då kan vi definiera \(
    F(\io) = 
\) FT för \(
    f
\).

Laplace kan definieras i vissa fall när FT inte existerar, ex.\ \(
    f(t) = e^t u(t)
\). Då är FT \(
    \approx \int_0^\infty e^t e^{-\io t} \dd t
\). \(
    \lap(f)(s) = \inv{s-1}, \Re{s} > 1
\).

För en kausal LTI \(
    y=h*x
\), om vi stoppar in \(
    x(t) = e^{st} 
\), ger \(
    y(t) = h * e^{st}
         = \int_0^\infty h(\tau) e^{s(t-\tau)} \dd \tau
         = e^{st} \int_0^\infty h(\tau) e^{-s\tau} \dd \tau
         = \lap(H)(s) x(t)
\).

Låt \(
    y' + ay = x
\) där \(
    y(0) 
\) är given.

Genom Laplace får vi \(
    \lap(y' + ay) = s Y(s) - y(0) + aY = X
        \implies Y(0) = \frac{X+y(0)}{s+a} 
            = \lap(e^{-at}u(t))\cdot (X+y(0))
            = \lap(e^{-at}u(t) * (x + y(0) \delta(t)))
\) vilket om \(
    \lap
\) är inverterbar ger att \(
    y(t) = e^{-at} u(t) * (x(t) + y(0) \delta(t))
         = y(0) e^{-at} u(t) + \int_0^t e^{-a\tau} x(t-\tau) \dd \tau
\).

\begin{ex}
    Låt \(
        f(t) = \cos(at) u(t)
    \) där \(
        a \in \realn
    \).

    \(
        f(t) = \frac{e^{iat} + e^{-iat}}{2} u(t) 
    \). Då är \(
        F(s) = \half (\lap(e^{iat} u(t)) + \lap(e^{-iat} u(t)))
             = \half( \inv{s-ia} + \inv{s+ia} ) 
             = \frac{s}{s^2+a^2} 
    \).
\end{ex}

\begin{ex}
    Låt \(
        f(t) = \sin(at), a \in \realn
    \). 

    \(
        \dv{t} \sin(at) = a \cos(at)
            \implies s \lap(\sin(at))(s) - \sin(a\cdot 0) 
                = a \cdot \frac{s}{s^2+a^2} 
            \implies s \lap(\sin(at))(s) = \frac{as}{a^2+s^2}
            \implies \lap(\sin(at))(s) = \frac{a}{a^2+s^2} 
    \) 
\end{ex}

Om \(
    \abs{f(t)} \leq e^{ct} 
\) och \(
    \sigma > c
\) är \(
    \int_0^\infty \abs{f(t) e^{-\sigma t}} \dd t < \infty
\) vilket ger att FT för \(
    f(t) e^{-\sigma t}
\) är väldefinierad. Utöver det bildar \(
    f(t) e^{-\sigma t}
\) ett Fouriertransformpar med \(
    F(\io + \sigma)
\).

Alltså är \(
    f(t) e^{-\sigma t} = \inv{2\pi} \infint F(\io + \sigma) e^{\io t} \dd \omega
        \implies f(t) = \frac{ e^{\sigma t} }{2\pi} \infint F(\io + \sigma) e^{\io t} \dd \omega
        = \{s = \io + \sigma\}
        = \inv{2\pi i} \int_{\sigma-\io}^{\sigma+\io} F(s) e^{st} \dd s
\) vilket kallas för \emph{Bromwichintegralen}.

\begin{sats}[Likhet av två signaler och dess två Laplacetransformer]
    Om \(
        f_1 
    \) och \(
        f_2
    \) är två kausala signaler med \(
        F_1 = F_2 
    \) är \(
        f_1 = f_2
    \).
\end{sats}

\begin{ex}
    Vi vill finna den signal \(
        f(t)
    \) så att \(
        \lap(f)(s) = \frac{s+1}{s^2+2} 
            = \frac{s}{s^2} + \frac{1}{s^2+2} 
            = \frac{s}{s^2+{\sqrt{2}}^2} + \inv{\sqrt{2}} \frac{\sqrt{2}}{s^2 + {\sqrt{2}}^2}
            = \lap(\cos(\sqrt{2} t)) + \lap(\inv{\sqrt{2}} \sin(\sqrt{2} t)) 
            \implies f(t) = \cos(\sqrt{2} t) + \inv{\sqrt{2}} \sin(\sqrt{2} t)
    \).
\end{ex}

\begin{ex}
    Anta att insignalen \(
        x
    \) uppfyller \(
        x(0) = 0
    \). Vi ska skriva om \(
        \left\{\begin{matrix}
            y'' + 2y + x'+x \\ 
            y(0) = y'(0) = 0
        \end{matrix}\right.
    \) på formen \(
        y = h*x
    \).

    Laplacetransform ger \(
        s^2 Y + 2Y = sX + X
        \iff
        Y = \frac{s+1}{s^2+2} X
        \iff 
        y = (\cos(\sqrt{2} t) + \inv{\sqrt{2}} \sin(\sqrt{2} t)) * x
    \).
\end{ex}

\begin{sats}[Slutvärdessatsen]
    Låt \(
        f
    \) vara en kausal och begränsad signal, d.v.s.\ \(
        \exists c > 0 
    \) s.a.\ \(
        \abs{f(t)} \leq c
    \).

    Satsen är att \(
        f(0+) = \lim_{\epsilon \to 0+} f(\epsilon) = \lim_{s \to \infty} s F(s)
    \) och \(
        \lim_{t \to \infty} f(t) = \lim_{s\to 0}sF(s)
    \).

    \begin{proof}
        Om \(
            s > 0
        \) är \(
            s F(s) = s \int_0^\infty f(t) e^{-st} \dd t 
                = \{u = st\} 
                = \int_0^\infty f(\frac{u}{s} ) e^{-u} \dd u
                \implies sF(s) = \lim_{s\to 0} \int_0^\infty f(\frac{u}{s} ) e^{-u} \dd u
                = \int_0^\infty \lim_{t\to \infty} f(t) e^{-u} \dd u
                = \lim_{t\to \infty} f(t) \int_0^\infty e^{-u} \dd u
                = \lim_{t\to \infty} f(t)
        \).
    \end{proof}
\end{sats}

\end{document}