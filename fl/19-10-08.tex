\documentclass[a4paper]{article}
\usepackage{../flpack}

\begin{document}

\providecommand\fname{}
\renewcommand\fname{19-10-08}

\subsection{Mer DTFT och DFT}
\begin{ex}
    Beräkna DTFT för signalen \(
        x[n] = \alpha^n u[n]
    \) för något \(
        0 < \alpha < 1
    \).

    \begin{align*}
        X(e^{\iO}) &= \sum_{n=0}^\infty \alpha^n e^{-\iO n} \\
            &= \sum_{n=0}^\infty {(\alpha e^{-\iO})}^n \\
            &= \{ \text{Geometrisk summa} \} \\
            &= \inv{ 1 - \alpha e^{-\iO} } \\
            &= \inv{ 1 - \alpha ( \cos(\Omega) - i \sin(\Omega) ) } \\
            &= \inv{ 1 - \alpha \cos(\Omega) + i \alpha \sin(\Omega) }
    \end{align*}

    \begin{align*}
        \abs{X(e^{\iO})} &= \inv{ \sqrt{ \abs{1- \alpha \cos \Omega }^2 + \abs{\alpha \sin \Omega}^2 } } \\
            &= \inv{ \left( 1-2\alpha \cos \Omega + \alpha^2 \cos^2(\Omega) + \alpha^2 \sin \Omega \right)^{\half} } \\
            &= \inv{ \sqrt{1 + \alpha^2 - 2 \alpha \cos \Omega} }
    \end{align*}

    \begin{align*}
        \arg{X(e^{\iO})} &= - \arctan( \frac{\alpha \sin \Omega}{1 - \alpha \cos \Omega} )
    \end{align*}
\end{ex}

För motsvarade kontinuerliga signal \(
    x(t) = e^{-iat}u(t)
\) har vi Fouriertransformen \(
    X(\io) = \inv{a-\io}
\), med \(
    \abs{X(\io)} = \inv{\sqrt{a^2+\omega^2}}
\) och \(
    \arg{X(\io)} = - \arctan{\frac{\omega}{a}}
\). Vi ser att \(
    \abs{X(\io)} \) upprepas periodiskt i \(
        X(e^{\iO})
\).

Genom att välja lämplig samplingsfrekvens kan effekten av aliassing minskas 
och \(
    X(e^\iO)
\) återfinns, i princip, i \(
    X(e^\iO)
\) på intervallet \(
    \pi < \Omega < \pi
\).

För numeriska beräkningar är det opraktiskt med kontinuerliga funktioner
såsåom \(
    X(e^{\iO})
\). Vad gör vi? Beräkna \(
    X(e^\iO)
\) endast för vissa frekvenser i intervallet \(
    0 \leq \Omega \leq 2\pi
\). Vi väljer ett \(
    \Omega_k = \frac{2\pi}{N} k
\) där \(
    k = 0, 1, \dots, N-1
\). Vi landar nu i det som kallas DFT (Diskret Fouriertransform). Den används
flitigt och har stor praktisk betydelse. Ytterligare en viktig aspekt
är att vi inte kan summera till oändligheten och därmed inte ha hur långa signaler 
som helst. Vi måste välja ut en representativ del. \(
    x[n] 
\) där \(
    n = 0, 1, \dots,N_{\text{max}}-1
\). 

\subsection{Diskret Fouriertransform (DFT)}
En godtycklig icke-periodisk och diskret signal \(
    x[n]
\) har Fouriertransformen \(
    X(e^\iO) = \sum_{n=-\infty}^\infty x[n] e^{-\iO n}
\). \(
    X(e^\iO)
\) är kontinuerlig i \(
    \Omega
\) och \(
    2\pi
\)-periodisk i \(
    \Omega
\). 

\subsubsection{Syntesekvation/Invers DTFT}
\[
    x[n] = \inv{2\pi} \int_{2\pi} X(e^{\iO}) e^{\iO n} \dd \Omega
\]

Det är en superposition av bassignaler \(
    e^{\iO n} = \cos(\Omega n) + i \sin(\Omega n)
\) med vikter \(
    X(e^{\iO})
\). Rent praktiskt kan vi bara hantera signaler \(
    x[n]
\) med ändlig längd. Dessutom är det najs om även transformen \(
    X
\) är diskret vilket skulle göra det rimligt för datorer att beräkna DTFT.

En alternativ Fouerirerepresentation har därför utvecklats, DFT.
Det som är viktigt med den är att \(
    DFT\{ x[n] \}
\) är diskret och att DFT motsvarar \enquote{samplade} värden av \(
    X(e^\iO)
\) på intervallet \(
    0 \leq \Omega < 2\pi
\). 

Effektiva beräkningsalgoritmer för att beräkna DFT kallas FFT 
(Fast Fourier Transform). 

Nu till DFT! Vi har tillgång till (eller väljer) \(
    L
\) stycken värden (sampel) i vår signal \(
    x[n]
\) där \(
    n = 0, 1, \dots, L-1
\), men ofta har vi ju bara samplat en kontinuerlig signal \(
    x(nT)
\). Signalens DTFT är då \(
    X(e^\iO) = \sum_{n=-\infty}^\infty x[n] e^{-\iO n}
        = \sum_{n=0}^{L-1} x[n] e^{-\iO n}
\). Det är en ändlig summa, vilket är bra, men det finns fortfarande 
oändligt många \(
    \Omega
\) vilket är dumt. Låt oss ta \(
    N
\) stycket avläsningar/sampel av \(
    X(e^\iO)
\) och gör beräkningar för \(
    \Omega_k = \frac{2\pi}{N} k
\) där \(
    k = 0, 1, \dots, N-1
\). Beteckna dessa sampel med \(
    X[k] = X(e^\iO) 
\) där \( 
    \Omega = \frac{2\pi}{N} k 
\) vilket ger att \[
    X[k] 
        = \sum_{n=0}^{L-1} x[n] e^{-i \frac{2\pi}{N} k n}.
\] 

Man kan visa att \(
    x[n]
\) kan återskapas ur \(
    X[k]
\) om \(
    N \geq L
\). Då får man \[
    x[n] = \inv{N} \sum_{k=0}^{N-1} X[k] e^{i \frac{2\pi}{N} k n}
\] 

Vi har då ett transformpar \(
    x[n] \xleftrightarrow{\text{DFT} } X[k]
\).

Det är vanligt att vi låter \(
    L = N
\) och att de är en jämn tvåpotens, för det behöver FFT för att funka.

\subsubsection{Frekvenser hos en samplad signal}

Kom ihåg att \(
    T
\) är vårt sampelintervall, att \(
    \Omega = \omega \cdot T
\), att \(
    \omega
\) är den kontinuerliga signalens vinkelfrekvens i \si{\radian\per\second}. 
Då har \(
    \Omega
\) enheten \si{\radian}. \(
    \Omega_k = \frac{2\pi}{N} k
\) där \(
    k = 0, 1, \dots, N-1
\) vilket med \(
    \omega = \frac{\Omega}{T} 
\) ger att \(
    \omega_k = \frac{\Omega_k}{T} 
        = \frac{2\pi}{N} \inv{T} k
        = \frac{2\pi}{T} \frac{k}{N} 
        = \frac{\omega_s}{N} k
\) där \(
    \omega_s
\) är samplingsvinkelfrekvensen. Om vi uttrycker samplingsfrekvensen i 
\si{\hertz} som \(
    F_s
\) är \(
    \omega_s = 2\pi F_s
\) vilket ger att \(
    \omega_k = \frac{2\pi F_s}{N} k 
\) vilket återkommer i labben. 

\end{document}