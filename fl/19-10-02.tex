\documentclass[a4paper]{article}
\usepackage{../flpack}

\begin{document}

\providecommand\fname{}
\renewcommand\fname{19-10-02}

\subsection{Laplacetransformen tillämpad på LTI-system}
Anta att vi har ett kontinuerligt LTI-system där sambandet mellan in- och 
utsignal beskrivs av en differentialekvation \[
    a_N \dv[N]{y}{t} + a_{N-1} \dv[N-1]{y}{t} + \cdots + a_1 \dv{y}{t} + a_0 y
    = b_M \dv[M]{x}{t} + b_{M-1} \dv[M-1]{x}{t} + \cdots + b_1 \dv{x}{t} + b_0 x
\] alternativt som \[
    \sum_{k=0}^N a_k \dv[k]{y}{t} = \sum_{k=0}^M b_k \dv[k]{x}{t}.
\] 

Anta nu att systemet är i vila, d.v.s.\ alla begynnelsevärden är \(
    0
\). När vi Laplacetransformerar får vi då \(
    \sum_{k=0}^N a_k s^k Y(s) = \sum_{k=0}^M b_k s^k X(s)
\) och därmed \(
    Y(s)\sum_{k=0}^N a_k s^k = X(s) \sum_{k=0}^M b_k s^k 
        \implies
        \frac{Y(s)}{X(s)} \coloneqq H(s) = \frac{\sum_{k=0}^M b_k s^k}{\sum_{k=0}^N a_k s^k} 
        = {b_M s^M + b_{M-1}s^{M-1} + \cdots + b_1 s + b_0}%
        {a_N s^N + a_{N-1}s^{N-1} + \cdots + a_1 s + a_0}
        \coloneqq \frac{B(s)}{A(s)} 
\). Detta är en kvot mellan två polynom beroende på \(
    s
\), d.v.s.\ en rationell funktion. Den formen fås väldigt ofta från 
Laplacetransformen i ingenjörstillämpningar. Notera att \(
    B(s) \neq Y(s)
\) och \(
    A(s) \neq X(s)
\), även om det ser ut så. Polynomen kan även skrivas på faktoriserad form 
enligt \(
    H(s) = \frac{b_M \prod_{k=1}^M (s-c_k)}{a_N \prod_{k=1}^N (s-d_k)} 
         = \frac{b_M (s-c_1)(s-c_2)\cdots (s-c_M)}{a_N (s-d_1)(s-d_2)\cdots (s-d_N)} 
\) där \(
    c_k
\) är nollställen till \(
    H(s)
\) och rötter till täljaren med symbolen \(
    \circ
\). \(
    d_k
\) är poler till \(
    H(s)
\) och är rätter till nämnarpolynomet har symbolen \(
    \times
\). En graf för ett \(
    H(s)
\) kan se ut som följer: \f

Det innehållet all information om \(
    H(s)
\) förutom konstanten \(
    \frac{b_M}{a_N} 
\). Allmänt kan \(
    c_k 
\) och \(
    d_k
\) vara antingen reella eller komplexa. Om det finns något komplext \(
    c_k
\) eller \(
    d_k
\) måste dess konjugat också vara ett nollställe så länge koefficienterna
i polynomen är reella. Det kommer allt som oftast vara fallet för vanliga 
fysikaliska system. 

Låt följande beskriva ett LTI-system: 
\(
    x(t) \xleftrightarrow{\lap} X(s)
\), \( 
    \xleftrightarrow{\lap} Y(s) 
\) och \(
    h(t) \xleftrightarrow{\lap} H(s)
\). 

I tidsdomänen har vi för systemet \(
    y(t) = h(t) * x(t)
\), i frekvensdomänen \(
    Y(s) = H(s)X(s)
\). Då skapar vi kvoten \(
    H(s) = \frac{Y(s)}{X(s)} 
\) vilket vi även fick när vi Laplacetransformerade den generella 
differentialekvationen tidigare. Vi kallar \(
    H(s)
\) för systemets överföringsfunktion. \(
    \lap^{-1}(H(s)) = h(t)
\) är systemets impulssvar. För kausala system använder vi den enkelsidiga
Laplacetransformen. Det räcker i den här kursen. 

\subsection{Invers Laplacetransform}
Utgå från en kvot mellan våra polynom, exempelvis \(
    H(s) = \frac{B(s)}{A(s)} 
\). Anta att \(
    \text{grad}(B(s)) = M
\) och \(
    \text{grad}(A(s)) = N 
\). För fysikaliska system är alltid \(
    M \leq N
\). Om \(
    M = N
\) vet vi att \(
    H(s) = c_0 + \frac{\widetilde{B}(s)}{A(s)} 
\) där \(
    \text{grad}(\widetilde{B}(s)) = M-1
\). Sen partialbråksuppdelar vi \(
    \frac{\widetilde{B}(s)}{A(s)} 
\) om \(
    M=N
\) och \(
    \frac{B(s)}{A(s)} 
\) direkt om \(
    M < N
\). Resultatet av partialbråksuppdelningen ger en summa av två typer av 
termer beroende på om rötterna är komplexa eller reella. Varje term kan 
sedan inverstransformeras. 

Man kan skriva partialbråksuppdelningsansatsen \(
    \frac{As+B}{{(s+\alpha)}^2 + \omega^2} 
\) som \(
    \frac{A(s+\alpha) + B - A \alpha}{{(s+\alpha)}^2 + \omega^2}
    = A \frac{s+\alpha}{{(s+\alpha)}^2 + \omega^2} + \frac{B-A\alpha}{\omega} \cdot \frac{\omega}{{(s+\alpha)}^2 + \omega^2}  
\) vilket ger inverstransformen \(
    A \left( e^{-\alpha t} \cos(\omega t) + \frac{B-A\alpha}{\omega} e^{-\alpha t} \sin(\omega t) \right) u(t)
\). För att den ska vara konvergera måste \(
    \Re{s_1} = \Re{s_2} < 0
\).

För att ett kausalt system ska vara stabilt måste alla poler till \(
    H(s)
\) ligga i vänstra halvplanet. 

\end{document}