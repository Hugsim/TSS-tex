\documentclass[a4paper]{article}
\usepackage{../flpack}

\begin{document}

\providecommand\fname{}
\renewcommand\fname{19-09-06}

\subsection{LTI-system}
Ett vanligt sätt att karakterisera ett system är att ange dess utsignal
för en given och känd insignal. För insignalen \(
    x(t) = \delta(t)
\) blir utsignalen \(
    y(t) = h(t)
\). Det kallas för systemets \emph{impulssvar}. Motsvarande samband gäller 
för ett diskret system. Andra vanliga insignaler för att beskriva system är
\begin{align*}
    \text{In} &\implies \text{ut} \\
    \text{Enhetssteg} &\implies \text{Stegsvar} \\
    \text{Sinusformad signal med } \omega = \omega_0 &\implies \text{Frekvenssvar} 
\end{align*}

\subsection{Samband mellan insignal, utsignal och LTI-system (i tidsdomänen)}
\subsubsection{Diskret fall}
Anta att vi känner impulssvaret \(
    h[n]
\) till ett diskret LTI-system. 

Låt \(
    x[n]
\) vara en godtycklig diskret signal. 

Bilda \(
    x[n] \cdot \delta[n] = x[0] \dot \delta[n]
\) och därefter bilda \(
    x[n]\cdot \delta[n-k] = x[k] \delta[n-k]
\). Tydligen kan vi teckna \(
    x[n]
\) som en summa av viktade och skiftade enhetsimpulser.
Alltså är \(
    x[n] = \bs_{k = -\infty}^{\infty} x[k] \cdot \delta[n-k]
\).

För ett LTI-system gäller 
\begin{align*}
    \text{Insignal} &\implies \text{Utsignal} \\
    \delta[n] &\implies h[n] \\
    \delta[n-k] &\implies h[n-k] \\
    x[k] \cdot \delta[n-k] &\implies x[k]h[n-k] \\
    x[n] = \bs_{k = -\infty}^{\infty} x[k] \cdot \delta[n-k] &\implies y[n] = \bs_{k = -\infty}^{\infty} x[k] \cdot h[n-k]
\end{align*}
\(
    y[n]
\) ovan kallas för Faltningssumman. Förenklat skrivs \(
    y[n] = x[n] * h[n]
\). Med en variabelsubstitution kan man visa att \(
    x[n] * h[n] = h[n] * x[n] \implies \bs_{k = -\infty}^{\infty} x[k] \cdot h[n-k] = \bs_{k = -\infty}^{\infty} h[k] \cdot x[n-k]
\).

\subsubsection{Kontinuerligt fall}
Anta att vi känner impulssvaret \(
    h(n)
\) till ett kontinuerligt LTI-system. Låt också \(
    x(t)
\) vara en godtycklig (in)signal och låt \(
    \hat{x}(t) 
\) vara en approximation av \(
    x(t)
\) där \(
    \hat{x}(t)
\) är summan av pulserna \(
    x_{-1}, x_0, x_1, \dots
\) o.s.v.

Vi definierar en enhetspuls som \(
    \delta_\epsilon(t) = \inv{\epsilon} \text{ när } 0 \leq t < \epsilon \text{ och } 0 \text{ annars.} 
\) 

\f

Våra pulser kan vi nu teckna som \(
    \dots, x_{-1} = \delta_\epsilon(t+\epsilon)x(-\epsilon)\epsilon, 
    x_0 = \delta_\epsilon(t)x(0)\epsilon,
    x_1 = \delta_\epsilon(t-\epsilon)x(\epsilon)\epsilon, \dots
\) och \(
    \hat{x}(t) = \bs_{-\infty}^{\infty} \delta_\epsilon(t-k\epsilon)x(k\epsilon)\epsilon
\). Låt \(
    h_\epsilon(t) 
\) vara systemets utsignal för insignalen \(
    \delta_\epsilon(t)
\) (pulssvar). För ett LTI-system gäller då \begin{align*}
    \text{Insignal} &\implies \text{Utsignal} \\
    \delta_\epsilon(t) &\implies h_\epsilon(t) \\
    \delta_\epsilon(t-k\epsilon) &\implies h_\epsilon(t-k\epsilon) \\
    \delta_\epsilon(t-k\epsilon)x(k\epsilon)\epsilon &\implies h_\epsilon(t-k\epsilon)x(k\epsilon)\epsilon \\
    \sum_{k = -\infty}^{\infty} \delta_\epsilon(t-k\epsilon)x(k\epsilon)\epsilon = \hat{x}(t) &\implies \sum_{k = -\infty}^{\infty} h_\epsilon(t-k\epsilon)x(k\epsilon)\epsilon = \hat{y}(t)
\end{align*}

Låt \(
    \epsilon \to 0
\), då gäller \begin{align*}
    \delta_\epsilon(t) &\to \delta(t) \\
    h_\epsilon(t) &\to h(t) \\
    k\epsilon &\to \tau \text{ (En kontinuerlig variabel)} 
    \epsilon &\to \dd \tau \\
    \sum &\to \int \\
    \hat{x}(t) &\to x(t) \\
    \hat{y}(t) &\to y(t) 
\end{align*}

Vi får då \[
    y(t) = \int_{-\infty}^{\infty} h(t-\tau)x(\tau)\dd \tau
\] vilket kallas faltningsintegralen. Förenklat skrivsätt är \(
    y(t) = h(t) * x(t)
\). Genom en variabelsubstitution kan man visa \(
    h(t) * x(t) = x(t) * h(t)
\), alltså att \(
    \int_{-\infty}^{\infty} h(t-\tau)x(\tau)\dd \tau = \int_{-\infty}^{\infty} x(t-\tau)h(\tau)\dd \tau
\) 

\aquote{Det är backe upp här och backe ner där.}

\subsection{Systemegenskaper kopplade till impulssvar}
\subsubsection{Kausalt LTI-system}
Diskret: \(
    h[k] = 0 \text{ för } k < 0
\) och därmed \(
    y[n] = \bs_{k = 0}^{\infty} h[k]x[n-k]
\). Motsvarande gäller för kontinuerliga system: \(
    y(n) = \int_0^\infty h(\tau)x(t-\tau) \dd \tau
\) 

\subsubsection{Stabilt LTI-system}
Diskret: Anta \(
    \forall n: \abs{x[n]} \leq M_x < \infty
\) d.v.s.\ att insignalen är begränsad. Utifrån det kan vi resonera att \(
    \abs{y[n]} = \abs{\bs_{k = -\infty}^{\infty} h[k]x[n-k]}
\) och eftersom \(
    \abs{a+b} \leq \abs{a} + \abs{b}
\) gäller \(
    \abs{y[n]} \leq \bs_{k = -\infty}^{\infty} \abs{h[k]x[n-k]}
\) och än en gång, eftersom \(
    \abs{ab} = \abs{a}\cdot\abs{b}
\) får vi \(
    \abs{y[n]} \leq \bs_{k = -\infty}^{\infty} \abs{h[k]}\abs{x[n-k]}
\) men eftersom \(
    \abs{x[k]} \leq M_x
\) vet vi att \(
    \abs{y[n]} \leq M_x \bs_{k = -\infty}^{\infty} \abs{h[k]}
\). Med kravet på stabila system att \(
    \forall n: \abs{y[n]} < \infty 
\) följer villkoret \(
    \bs_{-\infty}^{\infty} \abs{h[k]} < \infty
\). Det kallar man att impulssvaret är absolutsummerbart. 
Samma gäller för kontinuerliga system, då får man villkoret \(
    \int_{\infty}^\infty \abs{h(\tau)}\dd \tau < \infty
\) och det kallas för att impulssvaret är absolutintegrerbart.

\end{document}