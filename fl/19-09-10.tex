\documentclass[a4paper]{article}
\usepackage{../flpack}

\begin{document}

\providecommand\fname{}
\renewcommand\fname{19-09-10}

\subsection{Diff.\ ekvationer}
Vi kommer att titta på (LTIC-)system som beskrivs med \(
    Q(D)y = P(D)x
\) eller \(
    Q[E]y = P[E]x
\) i det diskreta fallet där \(
    D = \dv{t}
\) och \(
    Ef[k] = f[k+1]
\). Systemen är kausala, d.v.s.\ alltid \(
    = 0 \text{ för } t < 0 \text{ eller } k < 0
\). Alltså har vi ett begynnelsetillstånd. Den allmänna lösningen är \(
    y = y_0 + y_i
\) där \(
    y_0
\) är zerostate och \(
    y_i
\) är zeroinput. Vanligtvis är \(
    y_0 = h * x
\), d.v.s.\ någon slags faltning. 

Notation: \(
    Q(D) = D^m + a_{m-1}D^{m-1}+\cdots+a_1D+a_0
\) där \(
    m = ord(Q)
\) 

\subsubsection{Kontinuerliga fallet}
\(
    Q(D)y = P(d)x
\) och \(
    y^{(k)}(0) 
\) är givna för \(
    k=0, \dots, \text{ord}(Q)-1 
\).

Först löser vi \(
    Q(D)y_0 = P(D)f
\) för \(
    y^{(k)}(0) = 0
\). Sedan löser vi \(
    Q(D)y_i = 0
\) givet \(
    y_i^{(k)}(0) = y^{(k)}(0)
\). Detta är partikulär- och homogenlösningar i \enquote{vanliga} termer.
\(
    y_0 = h*x
\) där \(
    h
\) är impulssvaret.

\begin{ex}
    Anta \(
        y' + ay = x
    \), \(
        y(0) = b
    \), \(
        Q(D) = D+a
    \) och \(
        P(D) = 1
    \). 
    
    Eftersom systemet är tidsinvariant är \(
        a
    \) en konstant. Först löser vi \(
        y'_0 +ay_0 = x
    \) givet \(
        y_0(0)= 0
    \). Sedan komemr vi att lösa \(
        y'_i +ay_i = 0
    \) givet \(
        y_i(0)= b
    \). Vi använder integrerande faktor, d.v.s.\ \(
        (e^gy)' = (y'+g'y)e^g
    \) men eftersom vi vet\(
        a
    \) har vi \(
        g' = a \implies g = at
    \). Alltså är \(
        e^{at}(y'_0+ay_0) = (e^{at}y_0)'
    \) och därmed \(
        (e^{at}y_0)' = xe^{at} \implies e^{at}y_0(t) = e^{a0}y_0(0) + \bi_0^t x(s)e^{as}\dd s 
        = \bi_0^t x(s)e^{as}\dd s \implies y_0(t) = \bi_0^t x(s)e^{a(s-t)}\dd s
        = \bi_0^t x(s)e^{-a(t-s)}\dd s = x * h_{-a}(t) 
    \) där \(
        h_{-a}(t) = e^{at}u(t)
    \). Alltså \(
        y_0 = x * h_{-a}(t)
    \). Nu kollar vi om \(
        y_0
    \) är en lösning vårt inledande problem. Först kollar vi om \(
        y_0(0) = 0
    \) och det är fine eftersom det då blir en integral från \(
        0 
    \) till \(
        0
    \), vilket är \(
        0
    \). Nu kollar vi om \(
        y'_0 + a y_0 = x
    \). \(
        y'_0 = x(t) e^{-a(t-t)} - a \bi_0^t x(s) e^{-a(t-s)}\dd s = x(t) - a y_0(t)
    \) 


    Nu löser vi för \(
        y_i
    \). \(
        y'_i + a y_i = 0 \iff (e^{at} y_i) = 0 \iff e^{at}y_i(t) = y_i(0) = b
        \iff y_i(t) = be^{-at}
    \). 
    
    Alltså är \(
        y(t) = be^{-at} + x * h_{-a}(t)
    \) 
\end{ex}

\dquote{Det verkar trivialt att \(
    t-t = 0
\)  men det är det som räddar oss.}{Mattesnubben}

\subsubsection{Stabilitet i kausala fall}
Om \(
    \bi_0^\infty \abs{h(t)} \dd t < \infty 
\). \(
    h(t) = h_{-a}(t) = e^{-at}u(t) 
\). \(
    \bi_0^\infty \abs{h(t)} \dd t = \bi_0^\infty e^{-\Re{a}t} 
\) vilket är ändligt om \(
    \Re{a} > 0
\) och oändligt annars.

\begin{ex}
    Lös \(
        y'' + a_1y' + a_0y = x
    \) där \(
        y(0)
    \) och \(
        y'(0)
    \) givna. Vi vet att \(
        P(D) = 1
    \) och \(
        Q(D) = D^2+a_1D+a_0
    \).

    Vi börjar med att lösa ekvationen \(
        Q(\lambda) = \lambda^2+a_1\lambda+a_0 = 0
    \). Lösningarna till \(
        Q(\lambda) = 0 
    \) är \(
        \lambda_\pm = - \bif{a_1}{2} \pm \sqrt{\bif{a_1^2}{4} - a_0} \implies 
        Q(\lambda) = (\lambda - \lambda_+)(\lambda-\lambda_-)
    \). 

    Några exempel, om \(
        a_1 = 0
    \) får vi \(
        \lambda_\pm = \pm \sqrt{-a_0} \implies \lambda^2+a_0 = (\lambda - \sqrt{-a_0})(\lambda + \sqrt{-a_0})
    \), om \(
        a_0 = 0
    \) får vi \(
        \lambda_\pm = 0 
    \) eller \(
        -a_1 \implies \lambda(\lambda2+a_1)
    \).

    Vi börjar med att lösa zeroinput-lösningen. Observera att \(
        D^n(e^{\lambda t}) = \lambda^n e^{\lambda t} \implies 
        Q(D)e^{\lambda t} = Q(\lambda)e^{\lambda t} \implies
        Q(D)(\mu_1 e^{\lambda_+ t} + \mu_2 e^{\lambda_- t}) = 0
    \). Om \(
        \lambda_+ \neq \lambda_-
    \) hittar vi \(
        y_i
    \) genom ansatsen \(
        y_i(t) = \mu_1 e^{\lambda_+ t} + \mu_2 e^{\lambda_- t}
    \) så att \(
        y_i(0) = \mu_1 + \mu_2
    \) och \(
        y'_i (0) = \mu_1\lambda_+ + \mu_2 \lambda_-
    \).

    Observera nu också att \(
        Q(D)(te^{\lambda t}) = Q'(\lambda)e^{\lambda t} + Q(\lambda) te^{\lambda t}
    \). Om \(
        \lambda_+ = \lambda_-
    \), d.v.s.\ att \(
        Q(\lambda) = (\lambda-\lambda_+)^2
    \) gäller \(
        Q(D)(\mu_1 e^{\lambda_+ t} + \mu_2 te^{\lambda_+ t}) = 0
    \). Ansätt \(
        y_i(t) = \mu_1 e^{\lambda_+ t} + \mu_2 te^{\lambda_- t} 
    \) och finn \(
        \mu_1 \text{ och } \mu_2
    \) från \(
        y_i(0) = \mu_1
    \) och \(
        y'_i(0) = \lambda_+\mu_++\mu_2
    \).

    Nu löser vi zerostate:
    \(
        Q(D) = (D-\lambda_+)(D-\lambda_-)
    \). Notera att \(
        D^n(h*f) = (D^nh) * f
    \), vilket går att bevisa relativt lätt. Eftersom faltning är kommutativt 
    kan vi även derivera \(
        f
    \).

    \(
        Q(D)(h*f) = (Q(D)h) * f
    \). Ansätt \(
        h = h_{\lambda_+} * h_{\lambda_-}
    \). 

    Ansätt också \(
        y_0 = h * f
    \), då är \(
        (D-\lambda_- )y_0 = h_{\lambda_+}*f
    \) och \(
        h_{\lambda_+} * f(0) = 0
    \). 

    Jag orkade inte anteckna här, massa steg typ. Det blir enklare med 
    Laplace.
\end{ex}

\dquote{Vad kom \(
    \mu
\) ifrån?'' \\ ''Från ovan!}{Person som frågade och mattesnubben}

\dquote{Om vi nu delar den här skiten.}{Mattesnubben}

\subsection{Diskreta fallet}
I det diskreta fallet har vi ekvationen \(
    Q(E)y = P(E)x
\) där \(
    E(f[k]) = f[k+1]
\) och har \(
    y[-k]
\) givna för \(
    k = 0, \dots, ord(Q)-1
\). Funkar annars på samma sätt förutom att \(
    D 
\) ersätts med \(
    E
\) och \(
    y(0)
\) med \(
    y[-k]
\). 

I kontinuerliga fall har vi \(
    D(e^\lambda t) = \lambda e^{\lambda t}
\) som genom \(
    e_a[k] = a^k
\) motsvaras av \(
    E(e_a[k]) = e_a[k+1] = a^{k+1} = a \cdot a^k = a e_a
\).

\begin{ex}
    Vi löser \(
        y[k+1] + ay[k] = x[k]
    \) där \(
        y[0] 
    \) är given.

    Vi har \(
        Q(E) = E+a
    \). Vi börjar med att lösa \(
        y_i
    \), d.v.s.\ \(
        Q(E)y_i = 0
    \) och \(
        y_i[0] = y[0]
    \). Ansätt \(
        y_i = ce_{-a}
    \) så att \(
        (E+a)e_{-a} = 0
    \). Då har vi \(
        y_i[0] = ce_{-a}[0] = c = y[0]
    \) och därmed \(
        y_i[k] = y_i[0]a^k
    \).

    Nu löser vi för \(
        y_0
    \), då börjar vi med att ansätta \(
        h[k] = h_{-a}[k] = e_a[k]u[k]
    \) och \(
        y_0 = h * x
    \). 

    \(
        E(h*x[m]) = \bs_{l=1}^{m+1} h[l]x[m+1-l] = \bs_{l=1}^m h[l+1]x[m-l] + h[1]x[m]
    \) eftersom \(
        h*x = \text{ (om båda är kausala) } = \bs_{l=0}^m h[l]x[m-l]
        = \bs_{l=1}^m h[l]x[m-l]
    \). Allt som allt gäller \(
        E(h*x) = ah * x + ax
    \). Alltså \(
        y_0 = \inv{a} h* x
    \) vilket löser det ville innan.
\end{ex}

\dquote{Det kommer bara bli blodsspillan om vi har k där}{Mattesnubben}
\dquote{Vad gör vi nu, vad gör vi här, vem är jag?}{Mattesnubben}
\dquote{Nu börjar det lukta fågel}{Mattesnubben}
\dquote{När jag var i Tyskland var jag tvungen att dra bajsskämt för att det skulle gå, men här verkar det funka ändå.}{Mattesnubben}
\dquote{Fourier-serier är något extremt vackert, tårar kommer att fällas.}{Mattesnubben}

\end{document}