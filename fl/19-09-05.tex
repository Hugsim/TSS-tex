\documentclass[a4paper]{article}
\usepackage{../flpack}

\begin{document}

\providecommand\fname{}
\renewcommand\fname{19-09-05}

\subsection{Fler signaler}
\subsubsection{Enhetssteg (kontinuerligt)}
Betecknas \(
    u(t) = 1 \text{ om } t \geq 0, 0 \text{ annars} 
\).

\f

Den används oftast ihop med en generell signal \(
    x(t)\cdot u(t)
\) så att den får signalvärden \(
    0
\) när \(
    t < 0
\).

\subsubsection{Enhetsimpuls}
Tekniskt sett inte en \enquote{vanlig} funktion, det är snarare en distribution eftersom 
den inte har väldefinierade amplitudvärden för alla invärden.

\begin{defn}
    \(
        \delta(t) = 0 \text{ för } t \neq 0
    \) men \(
        \bi_{-\infty}^{\infty} \delta(t) \dd t = 1
    \) 
\end{defn}

Möjlig grafisk beskrivning: \f

Det är en \enquote{oändligt kort} signal med en \enquote{oändligt hög} 
amplitud. Amplituden vid \(
    t = 0
\) är inte begränsad. Vår grafiska notation är följande: \f

Enhetsimpulsen definieras utifrån sina egenskaper.

Låt \(
    f(t)
\) vara en godtycklig funktion (signal) som är kontinuerlig vid \(
    t = t_0
\). Då är \(
    f(t)\delta(t-t_0) = f(t_0)\delta(t-t_0)
\). Vidare gäller 
\begin{align*}
    &\bi_{-\infty}^{\infty} f(t)\delta(t-t_0)\dd t \\
    =& \bi_{-\infty}^{\infty} f(t_0)\delta(t-t_0)\dd t\\
    =& f(t_0) \bi_{-\infty}^{\infty} \delta(t-t_0)\dd t\\
    =& f(t_0)\text{.} 
\end{align*}

\subsubsection{Samband mellan enhetssteg och enhetsimpuls}
\(
    u(t) = \bi_{-\infty}^{t} \delta(\tau)\dd \tau
\) och \(
    \delta(t) = \displaystyle\dv{u(t)}{t}
\).

\subsubsection{Diskreta varianter}
Motsvarande gäller för diskreta varianter av enhetssteg och enhetsimpuls.

\f

\begin{defn}
    Den diskreta enhetsimpulsen \(
        \delta [n] = 1 \text{ om } n = 0, 0 \text{ annars.} 
    \) 
\end{defn}

\f

\subsubsection{Samband mellan diskreta varianter}
\(
    \delta [n] = u[n] - u[n-1]
\) och \(
    u[n] = \bs_{k=-\infty}^{n} \delta[k]
\) 

\subsection{System}
En process där det finns en relation mellan orsak (insignal) och verkan 
(utsignal). Kan symboliseras som ett vanligt blockschema med en låda
och pilar. En matematisk ekvation kan användas för att beskriva systemet, 
ex.\ elektriska kretsar och mekaniska system. Oftast differentialekvationer.

\subsection{Systemegenskaper}
\subsubsection{Tidsinvarians}
För ett tidsinvariant system gäller 
\begin{align*}
    \text{Insignal} &\implies \text{Utsignal} \\
    x(t) &\implies y(t) \\
    x(t-t_0) &\implies y(t-t_0)
\end{align*}

\aquote{Nu är det så att jag använder språket svenska.}

\f 

Ett system är då tidsinvariant om \(
    y(t-t_0) = y_d(t)
\) och samma gäller för ett diskret system. 

\subsubsection{Linearitet}
För ett linjärt system gäller 
\begin{align*}
    \text{Insignal} &\implies \text{Utsignal} \\
    x(t) &\implies y(t) \\
    a\cdot x(t) &\implies a\cdot y(t), a \text{ konstant (Systemet homogent)}  \\
    x_1(t) &\implies y_1(t) \\
    x_2(t) &\implies y_2(t) \\
    x_1(t) + x_2(t) &\implies y_1(t) + y_2(t), \text{ (Systemet additivt)}  \\
    a_1x_1(t) + a_2x_2(t)+\dots &\implies a_1y_1(t)+a_2y_2(t)+\dots \text{ (Kallas superposition)} 
\end{align*}
Om ett system är homogent och additivt är det linjärt.

\subsubsection{Stabilitet}
Ett system är stabilt om en begränsad insignal ger en begränsad utsignal. 
På engelska BIBO (Bounded input bounded output).

\(
    \forall t \abs{x(t)} < M_x < \infty \implies \abs{y(t)} < M_y < \infty
\) 

\subsubsection{Kausalitet}
Ett system är kausalt om utsignalen \(
    y(t)
\) endast beror på samtida och/eller tidigare värden på insignalen \(
    x(t)
\). Alla fysikaliska system är kausala om \(
    t 
\) är tid.

\subsubsection{Minne/Dynamiskt system}
Ett system har minne om dess utsignal vid tidpunkten \(
    t_0, y(t_0)
\), beror på fler insignaler än bara \(
    x(t_0)
\). 

\begin{ex}[Spänning över en kondensator]
    Insignalen är strömmen genom kondensatorn \(
        i(t)
    \) och utsignalen är spänningen \(
        v(t)
    \). Då är \(
        v(t) = \inv(C) \bi_0^t i(\tau) \dd \tau + v(0)
    \). \(
        v(t)
    \) beror på tidigare värden, alltså är systemet dynamiskt/har minne.
\end{ex}

\subsubsection{Minneslöshet/Statiskt system}
\begin{ex}[Spänning över en resistans]
    In- och utsignaler som förra exemplet, då är \(
        v(t) = R \cdot i(t)
    \). Eftersom det bara beror på \(
        i(t)
    \) och inga andra \(
        i
    \) är systemet minneslöst/statiskt.
\end{ex}

Alla dessa egenskaper gäller även för diskreta system. 

\begin{ex}[Diskret exempel]
    \[
        y[n] = n \cdot x[n]
    \]

    Vi ser om det är tidsinvariant. 
    \f

    Inte tidsinvariant eftersom \(
        y_d[n] \neq y[n-n_0]
    \).

    Kollar om det är linjärt:
    \begin{align*}
        \text{Insignal} &\implies \text{Utsignal} \\
        x_1[n] &\implies y_1[n] = n \cdot x_1[n] \\
        x_2[n] &\implies y_2[n] = n \cdot x_2[n] \\
        x_3[n] = a_1x_1[n] + a_2x_2[n] &\implies y_3[n] = n\cdot x_3[n]
        = n(a_1x_1[n] + a_2x_2[n]) = a_1nx_1[n]+a_2nx_2[n] = a_1y_1[n] + a_2y_2[n]
    \end{align*}
    Alltså är det linjärt.

    Det är inte stabilt eftersom \(
        y[n] = n\cdot x[n]
    \) inte är begränsat ty \(
        n
    \) inte är begränsat. Det är dock kausalt, vilket inses lätt.
\end{ex}

\end{document}