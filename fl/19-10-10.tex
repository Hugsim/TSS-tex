\documentclass[a4paper]{article}
\usepackage{../flpack}

\begin{document}

\providecommand\fname{}
\renewcommand\fname{19-10-10}

\subsection{Transformer i allmänhet}
Kom ihåg att Fourierkoefficienterna för en Fourierserie ges av \[
    c_k(f) = \inv{T} \int_0^T f(t) e^{-jk\omega_0 t} \dd t.
\] 

Kom också ihåg att för periodiska \(
    y
\), \(
    h
\) och \(
    x
\) som har sambandet \(
    y(t) = h*x(t)
\) är \(
    c_k(y) = c_k(h) \cdot c_k(x)
\) för alla \(
    k
\).

Kom utöver det också ihåg att \(
    c_k(x') = i k c_k(x)
\).

En transform är något som tar en signal \(
    f
\) till en ny beskrivning för \(
    f
\).

\subsection{z-transform}
Låt \(
    x[k]
\) vara en diskret signal. Den ska vi \enquote{skicka på en funktion} av en 
komplex variabel \(
    X(z) \coloneqq \sum_{k=-\infty}^\infty x[k] z^{-k}
\). 

Om \(
    f
\) är en \(
    T
\)-periodisk kontinuerlig signal och \(
    x[k] = c_k(f)
\) så är \(
    X(z) = \sum_{k=-\infty}^\infty c_k(f) z^{-k}
\). Om vi väljer \(
    z = e^{-i \frac{2\pi}{T} t}
\) så är \(
    X(e^{-i \frac{2\pi}{T} t}) = f(t)
\). Därav är z-transformen i någon mån en inverstransform till Fourierserier.

\begin{påst}
    Låt \(
        x[k]
    \) vara en diskret signal. Anta att det finns tal \(
        0 < c_0 < c_1 < \infty 
    \) så att \(
        \abs{x[k]} \leq c_0^k 
    \) för \(
        k >> 0
    \) och \(
        \abs{x[k]} \leq c_1^k 
    \) för \(
        k << 0
    \). Då är \(
        X(z)
    \) väldefinierad och absolutkonvergent för \(
        c_0 < \abs{z} < c_1
    \).

    \begin{proof}
        Idén för beviset är att visa att \(
            \abs{X(z)} \leq \sum_{k=-\infty}^\infty \abs{x[k]} \abs{z}^k
        \) konvergerar.
    \end{proof}
\end{påst}

Vi kommer att anta att \(
    x[k]=0
\) för \(
    k < 0
\), d.v.s.\ det kausala fallet. 

Då är \(
    X(z) = \sum_{k=0}^\infty x[k]z^{-k}
\). Om \(
    \abs{x[k]} \leq c^k
\) för \(
    k >> 0
\) gäller att \(
    X(z)
\) finns för \(
    \abs{z} > c
\).

\begin{ex}
    Låt \(
        x[k] = \delta(k-l)
    \) där \(
        l \geq 0
    \).

    Då är \(
        X(z) = \sum_{n=0}^\infty \delta(n-l) z^{-n}
            = z^{-l}
    \). 
\end{ex}

\begin{ex}
    Låt \(
        x_\gamma[k] = \gamma^k u[k]
    \) där \(
        \gamma \in \compn \setminus \{0\}
    \).

    Då är \(
        X_\gamma(z) = \sum_{k=0}^\infty \gamma^k z^{-k}
            = \sum_{k=0}^\infty \left(\frac{\gamma}{z} \right)^k
            = \frac{1}{1-\frac{\gamma}{z}} 
    \) vilket bara konvergerar om \(
        \frac{\gamma}{z} < 1
    \) enligt påminnelsen nedan, vilket ger att den konvergerar om \(
        \abs{z} > \abs{\gamma}
    \). Då får vi \(
        \frac{1}{1-\frac{\gamma}{z}} = \frac{z}{z-\gamma} 
    \).
\end{ex}

\begin{påm}
    \(
        \frac{1}{1-w} = \sum_{k=0}^\infty w^k
    \) och \(
        \sum_{k=0}^N w^k = \frac{1-w^{N+1}}{1-w} 
    \). Den första konvergerar för \(
        \abs{w} < 1
    \).
\end{påm}

\begin{ex}
    Finn en diskret och kausal signal \(
        h_\gamma
    \) med \(
        X(z) = \frac{1}{z-\gamma} 
    \).

    Om vi tar \(
        x_\gamma
    \) från exemplet ovan får vi att \(
        X_\gamma(z) = \frac{z-\gamma+\gamma}{z-\gamma} 
            = 1 + \frac{\gamma}{z-\gamma} 
    \) vilket ger att \(
        h_\gamma = \inv{\gamma} \left( x_\gamma - \delta_0 \right)
    \), d.v.s.\ \(
        h_\gamma[k] = 
        \left\{\begin{matrix}
            \gamma^{k-1} & k > 0 \\ 
            0 & k = 0 
        \end{matrix}\right. 
        = x_\gamma[k-1]u[k]
    \).
\end{ex}

\subsection{Syntes för z-transform}
\begin{sats}[Syntes för z-transformen]
    \[
        x[k] = r^k \int_0^1 X(re^{2\pi i t}) e^{2\pi i k t} \dd t
    \] där \(
        r
    \) är något tal s.a.\ \(
        X(z) 
    \) är definierat för \(
        \abs{z} \geq r
    \).

    \begin{proof}
        \begin{align*}
            x[k] &= \int_0^1 X(re^{2\pi i t}) e^{2\pi i k t} \dd t \\
                 &= \int_0^1 \sum_{n=0}^\infty x[n] r^{-n} e^{-2\pi int} 
                    e^{2\pi ikt} \\
                 &= \sum_{n=0}^\infty x[n] r^{-n} \int_0^1 e^{2\pi ikt} \\
                 &= x[k]r^{-k}
        \end{align*}
        eftersom integralen är \(
            0 
        \) för \(
            k \neq n
        \) och \(
            1
        \) för \(
            k = n
        \).
    \end{proof}
\end{sats}

\subsection{Viktiga egenskaper för z-transformen}
Z-transformen är linjär, d.v.s.\ \(
    x_1 + \lambda x_2 \xleftrightarrow{z} X_1 + \lambda X_2
\).

För ett vänsterskifte, d.v.s.\ om vi låter \(
    x_s[k] = x[k+1] u[k]
\) så har vi transformparet \(
    x_s[k] = x[k+1] u[k] \xleftrightarrow{z} z(X(z) - x[0])
\).

Högerskifte, d.v.s.\ \(
    x_s[k] = x[k-1] 
\) ger transformparet \(
    x_s[k] = x[k-1] \xleftrightarrow{z} \frac{1}{z} X(z)
\).

Slutvärdessats för z-transformen säger att \(
    x[0] = \lim_{\abs{z}\to \infty} x[z]
\).

Motsvarande initialvärdessats säger att, givet att \(
    x[k] 
\) är begränsad och att \(
    \lim_{k \to \infty} x[k] 
\) existerar, då är \(
    \lim_{h\to\infty} x[k] = \lim_{z \to 1}(z-1)X(z)
\).

\begin{ex}
    Låt \(
        x[k] = u[k]
    \). Det är då transformpar med \(
        \frac{z}{z-1} 
    \). Vi vet då också att \(
        1 = \lim_{k\to\infty} u[k] = \lim_{z\to 1} (z-1) \frac{z}{z-1} 
    \).
\end{ex}

\(
    \gamma^k x[k]
\) bildar transformpar med \(
    X(\frac{z}{\gamma} )
\) eftersom \(
    \sum_{k=0}^\infty \gamma^k x[k] z^{-k} 
        = \sum_{k=0}^\infty \gamma^k x[k] \left(\frac{z}{\gamma}\right)^{-k} 
\).

\(
    kx[k]
\) bildar transformpar med \(
    -z \dv{z} X(z)
\), idén bakom är att \(
    -z \dv{z} X(z) = -z \cdot (-k z^{k-1}) = k z^{-k}
\).

\(
    x_1 * x_2 \xleftrightarrow{z} X_1X_2
\) eftersom \(
    \sum_{k=0}^\infty (x_1*x_2)[k] z^{-k}
        = \sum_{k=0}^\infty \sum_{l=0}^k x_1[k-l]x_2[l] z^{-k}
        = \sum_{m,l=0}^\infty x_1[m]x_2[l] z^{-(m+l)}
        = \sum_{m=0}^\infty x_1[m] z^{-m} \cdot \sum_{l=0}^\infty x_1[l] z^{-l}
\).

För en diskret och kausal LTI \(
    y = h*x \iff Y = HX
\).

\begin{ex}
    Låt \(
        x[k] = \gamma^k
    \). 
    
    Då är \(
        h*x[l] = \sum_{m} h[m] \gamma^{l-m}
            = \gamma^l \sum_m h[m] \gamma^{-m}
            = \gamma^l H(\gamma)
            = H(\gamma) x[l]
    \).
\end{ex}

\begin{ex}
    Låt \(
        x[k] = \gamma^k \cos(\beta k) u[k]
    \).

    Vi börjar med att anta att \(
        \gamma = 1
    \). Då är \(
        x[k] = \half \left( e^{i\beta k} + e^{-i\beta k} \right) u[k]
             = \half \left( {(e^{i\beta})}^k + {(e^{-i\beta})}^k \right) u[k]
    \). Då bildar det transformpar med \(
        \inv{z} \left( \frac{z}{z-e^{i\beta k}} + \frac{z}{z-e^{-i\beta k}} \right)
    \). 

    Om \(
        \gamma \neq 1
    \) får vi att \(
        \gamma^k \cos(\beta k) u[k] 
    \) bildar transformpar med \(
        \half \left( \frac{{\frac{z}{\gamma}}}{{\frac{z}{\gamma}}-e^{i\beta k}} + \frac{{\frac{z}{\gamma}}}{{\frac{z}{\gamma}}-e^{-i\beta k}} \right)
            = \dots 
    \) Man kan göra mer algebra för att få något \enquote{finare} om man vill.
\end{ex}

\subsection{Differensekvationer}
Vi vill lösa \(
    y[k+1] + ay[k] = x[k]
\) med villkoret att \(
    k\geq 0
\) och att \(
    y[0]
\) är given. 

Notera att det är på formen \(
    Q(E)y = P(E)x
\) där \(
    E 
\) är vänsterskifte och \(
    Q(z) = z + a
\) och \(
    P(z) = 1
\).

\(
    Ey[k] = y[k+1]u[k] \xleftrightarrow{z} = z Y(z)-zy[0]
\) ger att \(
    Q(E)y[k] \xleftrightarrow{z} z Y(z) - zy[0] + a Y(z)
        = (z+a)Y(z)-zy[0]
        = Q(z)Y(z)-zy[0]
\).

\(
    Q(E)y = P(E)x 
        \iff Q(z)Y - zy[0] = X(z)
        \iff Y(z) = \inv{Q(z)} X(z) + \frac{z}{Q(z)} y[0]
            = \frac{1}{z+a} X(z) + \frac{z}{z+a} y[0]
        \iff y[k] = h_a*x[k] + (-a)^k y[0]
            = \left(\sum_{l=0}^k (-a)^{l-1} x[k-l]\right) + (-a)^k y[0]
\). Sista \(
    \iff
\)-steget görs genom z-transformstabeller. 
Resten lämnas som övning till läsaren.

\end{document}